%\input{tcilatex}
%\input{tcilatex}
%\input{tcilatex}
%\input{tcilatex}


\documentclass[12pt,english]{article}
%%%%%%%%%%%%%%%%%%%%%%%%%%%%%%%%%%%%%%%%%%%%%%%%%%%%%%%%%%%%%%%%%%%%%%%%%%%%%%%%%%%%%%%%%%%%%%%%%%%%%%%%%%%%%%%%%%%%%%%%%%%%%%%%%%%%%%%%%%%%%%%%%%%%%%%%%%%%%%%%%%%%%%%%%%%%%%%%%%%%%%%%%%%%%%%%%%%%%%%%%%%%%%%%%%%%%%%%%%%%%%%%%%%%%%%%%%%%%%%%%%%%%%%%%%%%
\usepackage[T1]{fontenc}
\usepackage[latin9]{inputenc}
\usepackage{babel}
\usepackage{amsfonts,amsmath,amssymb,amsfonts,graphicx,natbib}
\usepackage{pbox}
\usepackage[top=1.25in,left=1.25in,right=1.25in,bottom=1.25in]{geometry}
\usepackage{setspace}
\usepackage{hyperref}
\usepackage[para,online,flushleft]{threeparttable}
\usepackage{xcolor}

\setcounter{MaxMatrixCols}{10}
%TCIDATA{OutputFilter=LATEX.DLL}
%TCIDATA{Version=5.50.0.2960}
%TCIDATA{<META NAME="SaveForMode" CONTENT="1">}
%TCIDATA{BibliographyScheme=Manual}
%TCIDATA{LastRevised=Thursday, November 12, 2020 18:11:53}
%TCIDATA{<META NAME="GraphicsSave" CONTENT="32">}
%TCIDATA{Language=American English}

\onehalfspacing
%\input{tcilatex}
\begin{document}

\title{Skill-Biased Structural Change\thanks{%
We thank four anonymous referees, Dirk Krueger, Daron Acemoglu, David Dorn,
Chad Jones, Pete Klenow, Tommaso Porzio as well as seminar and conference
participants at the 2015 AEA Meetings, Arizona State University, the Canon
Institute for Global Studies, Chicago Fed, Paris School of Economics,
Philadelphia Fed, Pittsburgh, Stanford, University of Houston, USC, Wharton,
the World Bank, Yonsei University, UCL, the CEPR, and the Institute for New
Structural Economics (INSE) for useful comments.}}
\author{Francisco J. Buera \\
%EndAName
Washington University in St. Louis \and Joseph P. Kaboski \\
%EndAName
University of Notre Dame \and Richard Rogerson \\
%EndAName
Princeton University \and Juan I. Vizcaino \\
%EndAName
University of Nottingham}
\maketitle

\begin{abstract}
Using a broad panel of advanced economies we document that increases in GDP
per capita are associated with a systematic shift in the composition of
value added to sectors that are intensive in high-skill labor, a process we
label as skill-biased structural change. It follows that further development
in these economies leads to an increase in the relative demand for skilled
labor. We develop a quantitative two-sector model of this process as a
laboratory to assess the sources of the rise of the skill premium in the US
and a set of ten other advanced economies, over the period 1977 to 2005. For
the US, we find that the sector-specific skill neutral component of
technical change accounts for 18-24\% of the overall increase of the skill
premium due to technical change, and that the mechanism through which this
component of technical change affects the skill premium is via skill biased
structural change.
\end{abstract}

\bigskip \newpage

\section{Introduction}

The substantial increase in the wages of high-skilled workers relative to
low-skilled workers is one of the most prominent secular trends in the US
and other advanced economies. A large literature that seeks to isolate the
underlying driving forces and propagation mechanisms behind this trend has
consistently concluded that skill-biased technological change (SBTC) is a
quantitatively important driver of this increase.\footnote{%
Important early contributions to the literature on the skill premium that
stress skill-biased technical change include \citet{KatMur92}, %
\citet{BouJoh92}, \citet{MurWel92}, \citet{BBG94} and \citet{BBM98}. This is
not to say that SBTC is the \emph{only} factor at work, as the literature
has also highlighted the effect of other factors on overall wage inequality.
For example, \citet{DFL96} argue that labor market institutions such as
minimum wages and unionization have played an important role in shaping wage
inequality overall, \citet{FeeHan99} emphasize the role of offshoring, and %
\citet{ADH13} emphasize the role of trade.} In this paper we argue that a
distinct process -- which we label skill-biased \emph{structural} change --
has also played a quantitatively important role. We use the term skill
biased structural change to describe the \emph{systematic} reallocation of
sectoral value added shares toward relatively skill intensive industries
that accompanies the growth process among advanced economies and that is
driven by sector-specific skill-neutral technical change.

The economic intuition behind our finding is simple. If (as we show is
indeed the case in the next section) the process of development is
systematically associated with a shift in the composition of value added
toward sectors that are intensive in high-skill workers, then the relative
demand for high-skilled workers will increase, even if development is driven
by the skill-neutral component of technical change. This channel is absent
in analyses that adopt an aggregate production function, since in that case
the skill-neutral component of technical change has no effect on the
relative demand for high-skilled workers.

Our argument proceeds in four steps. We first develop a simple general
equilibrium model of structural change that incorporates an important role
for skill. To best highlight the shift in value added to relatively skill
intensive sectors, we study a two-sector model in which the two sectors are
distinguished by their intensity of skilled workers in production. We allow
for sector-specific technological change, and decompose technological change
in each sector into a skill-neutral component and a skill-biased component.
The skill-biased component captures technological change that affects
relative marginal products holding inputs fixed, while the skill neutral
component captures technical change that affects the amount of output
holding inputs fixed. This decomposition is of interest precisely because of
the fact that in one-sector models the skill premium depends only on the
skill-biased component and is independent of whether total output is
affected. We derive a log linear approximation for a special case of our
model and show that changes in the sector-specific skill-neutral component
of technical change can impact the skill premium, and that the lone
mechanism through which this happens is by reallocating activity across
sectors that differ in their skill intensity.

In the second step, we use the model to study the evolution of the skill
premium in the US economy between 1977 and 2005. We assume that the only
exogenous driving forces are technical change and changes in the relative
supply of skilled workers. We measure the change in the relative supply of
skill directly from the data. We show how the model can be used to infer
preference parameters and the components of technical change using data on
the growth in aggregate output, relative sectoral prices and the
distribution of sectoral value added, the skill premium and changes in
sectoral factor shares. Importantly, our calibrated model perfectly matches
the observed increase in the skill premium, which rises from $1.33$ to $1.88$
between 1977 and 2005.

In the third step we use our model to decompose the overall increase of the
skill premium into four components: one due to the change in the relative
supply of high-skill workers, a second due to skill-biased technical change,
a third due to sector-specific skill-neutral change, and a fourth term that
represents the interaction between the two types of technical change. If
there had been no technical change, our model predicts that the increase in
the relative supply of high-skill workers would have lowered the skill
premium to $0.87$, a drop of $46$ percentage points. It follows that
technical change created an increase in the skill premium of $101$
percentage points. In our benchmark specification, between $18$ and $24$
percent of this increase comes from changes in the sector-specific
skill-neutral component of technological change.\footnote{%
The range of estimates reflects the effect of varying the allocation of the
interaction term.}

The fourth and final step quantifies the mechanism through which the
sector-specific skill-neutral component of technical change affects the
skill premium. We show that this component drives the rise in the size of
the high skill-intensive sector; in fact, in its absence, the value added
share of the skill-intensive sector would have decreased modestly. The
sector-specific skill-neutral component of technical change is also the
dominant source of increases in output in our model. We conclude that
systematic changes in the composition of value added associated with the
process of development are an important mechanism in accounting for the rise
in the skill premium.

To assess the importance of skill-biased structural change more broadly we
repeat the analysis for a set of ten other OECD countries. While the
contribution of the sector-specific skill-neutral component of technical
change varies across countries, ranging from around $15$ percent to almost $%
50$ percent, the median for this sample is $23$ percent, very much in line
with our estimates for the US.

Our paper is related to many others in two large and distinct literatures,
one on SBTC and the skill premium and the other on structural
transformation. Important early contributions to the literature on the skill
premium include \citet{KatMur92}, \citet{BouJoh92}, \citet{MurWel92}, %
\citet{BBG94} and \citet{BBM98}. These papers emphasized the role of SBTC
because the increase in the skill premium occurred despite a large increase
in the relative supply of high-skill workers, and they could not identify
other factors that would lead to a large increase in the relative demand for
high-skill workers. In particular, while each of them noted compositional
changes in demand as a potentially important factor, none of them found this
channel to be of first order importance.

Relative to this literature, our contribution is fourfold. First, we analyze
the evolution of the skill premium in general equilibrium in a multi-sector
economy. Second, we decompose technological change into sector-specific
skill-biased and skill-neutral components and assess the contribution of
each component to the evolution of the skill premium. Third, we find a large
role for the skill-neutral component of technical change and show that the
key mechanism through which it affects the skill premium is via structural
change. Fourth, we link the driving forces of this structural change to the
process of development. In Section 6, we detail the key reasons that our
model based approach leads to a different conclusion than the shift-share
approach followed by \citet{KatMur92}.

An early contribution in the second literature is \citet{Bau67}, with more
recent contributions by \citet{KRX01} and \citet{NgaPis07}. (See %
\citet{HRV14} for a recent overview.) Relative to this literature our main
contribution is to introduce heterogeneity in worker skill levels into the
analysis and to organize industries by skill intensity rather than broad
sectors. \citet{CasCol01} is an early paper linking structural
transformation and human capital. Differently than us, they focus on the
movement of resources out of agriculture and into non-agriculture, and
assume that the non-agricultural sector uses only skilled labor.

Two closely related papers to ours are \citet{BueKab12} and \citet{Leo15}. %
\citet{BueKab12} also study the interaction between development and the
demand for skill both empirically and theoretically, but their primary
theoretical contribution is conceptual, building a somewhat abstract model
to illustrate the mechanism. Relative to them we make three contributions.
First, we document the empirical patterns for a larger set of countries and
along additional dimensions. Second, we develop a model that nests the
benchmark quantitative models used to study the skill premium and structural
change in isolation. Their model did not include sector-specific technical
progress, the driving force behind structural change in our model. Third, we
use the model to quantitatively assess the contribution of the
sector-specific, skill-neutral component of technical change.

\citet{Leo15} also considers how changes in primitives are propagated via
structural change to generate changes in the skill premium. But whereas we
focus on the effect of the skill-neutral component of technical change and
find significant effects, he focuses on changes in educational attainment
and finds relatively small effects. In Section 6 we show that our model
predicts similarly small effects for the change that he focuses on.

Our model structure is broadly similar to that of \citet{AceGue08}. Like us,
they study the relationship between development and structural change in a
model that features heterogeneity in factor intensities across sectors. But
whereas we focus on differential intensities of human capital, they focus on
differential intensities of physical capital. \citet{CraSot19} use a
framework similar to ours to show how reductions in trade costs affect
demand composition and the demand for skill. \citet{CMR18} incorporates
skill-biased technical change into a model of structural change. Their focus
is on labor market polarization and the role of gender differences.\footnote{%
\citet{NgaPet14} use a similar framework to show that compositional changes
in value added associated with development can explain part of the decrease
in the gender wage gap that has occurred in the US over time.}

An outline of the paper follows. Section 2 documents the prevalence of skill
biased structural change in a panel of advanced economies, and evidence
suggestive of the mechanisms that drive it. Section 3 presents our general
equilibrium model and characterizes the equilibrium. Section 4 shows how the
model can be used to account for the evolution of the US economy over the
period 1977 to 2005, and in particular how the data can be used to infer
preference parameters and the process of technical change. Section 5
presents our main results about the role of different driving forces on the
evolution of the skill premium, and the relation between the skill-neutral
component of technical change and structural change. Section 6 discusses the
relationship of our results to earlier results in the literature. Section 7
reports results for several sensitivity exercises. Section 8 extends the
analysis to a set of nine other countries and Section 9 concludes.

\section{Skill-Biased Structural Change: Empirics}

This section establishes skill-biased structural change as a robust feature
of the data for advanced economies. Using data for a broad panel of advanced
economies, we document a strong positive correlation in the time series
between the level of development in an economy, as measured by GDP per
capita, and the share of economic activity accounted for by the relatively
skill-intensive sector. The traditional structural change literature
documents patterns both in terms of output value added shares as well as
employment shares. Similarly, we find that this pattern holds whether we
measure the size of the skill-intensive sector in terms of its output
value-added share or its share of overall labor compensation. This
relationship is robust across countries, and in particular, the experience
of the US is very similar to the average pattern found in the data.

The structural change literature emphasizes two key mechanisms: relative
price effects and income effects. We document that both of these mechanisms
seem relevant for understanding the increasing size of the skill-intensive
sector. First, using cross-country data we document a strong positive
correlation in the time series between the level of development and the
price of the skill-intensive sector relative to other goods and services.
Once again, this relationship is robust across countries and the experience
of the US is similar to the average pattern.

Second, to document the potential importance of income effects we supplement
the aggregate time series panel analysis with evidence on cross-sectional
expenditure shares in the US economy. We show that the expenditure of higher
income households contains a higher share of value added from the
skill-intensive sector.

\subsection{Data Sources}

The facts reported in this section are based on several datasets. Sectoral
value-added shares, sectoral compensation shares and relative sectoral
prices come from the EUKLEMS Database (\textquotedblleft Basic
Table\textquotedblright ).\footnote{%
See \citet{OMaTim09}.} These data exist in comparable form for a panel of
countries over the years 1970-2005. The sectoral data are available at
roughly the 1 to 2-digit industry level. Our focus is on advanced economies'
growth experience, so following \citet{BueKab12}, we focus on the 15
countries with income per capita of at least 9,200 Gheary-Khamis 1990
international dollars at the beginning of the panel in 1970.\footnote{%
These countries are Australia, Austria, Belgium, Denmark, France, Germany,
Greece, Ireland, Italy, Japan, the Netherlands, Spain, Sweden, the United
Kingdom, and the United States. We exclude Luxembourg given its small size.
The U.S. data for value added go back to only 1977, while the Japan data go
back to only 1973.} Cross-country data for real (chain-weighted) GDP per
capita data is from the Penn World Tables 9.0.

Labor compensation data come from the EUKLEMS Labour Input Data. This
dataset reports the share of sectoral labor compensation that goes to
different skill groups. We define the high-skill group to be those with a
college degree or more and define the low-skill group to be all other
workers.

Our analysis of US micro data is based on the Consumer Expenditure Survey
(CEX), a cross-section data set on household expenditure. This dataset
reports household expenditure on final expenditure categories and not
value-added categories. To create consistent measures we map household
expenditure data through the input-output system to determine the
consumption shares of value added. We briefly sketch the steps of this
procedure here, and provide more details in the Online Appendix.

We start with the household level CEX data for the United States from 2012
and clean the data as in \citet{AguBil15}. We adapt a Bureau of Labor
Statistics mapping from disaggregated CEX categories to 76 NIPA Personal
Consumption Expenditure (PCE) categories and then utilize a Bureau of
Economic Analysis (BEA) mapping of these 76 PCE categories to 69
input-output industries that properly attributes the components going to
distribution margins (disaggregated transportation, retail, and wholesale
categories). Using the 2012 BEA input-output matrices, we can then infer the
quantity of value added of each industry embodied in the CEX expenditures.
In our empirical work we restrict ourselves to the primary interview sample,
respondents age 24-65 with complete income records, and each observation is
a household-quarter observation.

\subsection{Defining the Skill-Intensive Sector}

Both our empirical and theoretical analysis will focus on an aggregation of
sectors into two bins based on skill intensity: a high-skill intensive
sector and a low-skill intensive sector. An important first step is to
assign individual sectors to these two broad categories. Our primary metric
for assessing skill intensity is the share of total sectoral labor
compensation that goes to high-skill workers.\footnote{%
We have also considered the share of total hours or employment accounted for
high-skill workers. These alternatives are all highly correlated with our
baseline metric based on compensation shares and so do not suggest an
alternative ranking. See the Online Appendix for more details.} Importantly,
the ranking of sectors via this criterion is quite stable over time, so that
relative skill intensity can safely be viewed as fixed characteristic of a
sector over the period of our analysis.\footnote{%
See the Online Appendix for more details on this.}

Creating a binary characterization requires that one adopts a boundary
between the high- and low-skill intensive sectors. The benchmark results
that we report in this paper are based on defining the skill-intensive
sector as consisting of Education, Renting of Machinery and Equipment and
Other Business Activities, Financial Intermediation, and Health and Social
Work. These four sectors have both the highest average value for their
high-skill compensation shares as well as the highest average ranking in the
distribution of high-skill compensation shares. In terms of average ranks
over the 1977-2005 period, the above four sectors have values of 1.6, 2.4,
3.3 and 4.5 respectively. The next two highest are Chemicals and Chemical
Products (5.1) and Real Estate (5.8). In terms of average high-skill
compensation shares our four sectors have values of 0.753, 0.528, 0.506, and
0.466. The next two highest are Chemicals and Chemical Products (0.446), and
Real Estate (0.434). No other sector has an average above 0.40.

The Real Estate sector merits some discussion. Although this sector ranks
quite highly, we have chosen to exclude it from our benchmark definition of
the skill-intensive sector. Because this sector has very little employment,
its assignment is effectively inconsequential from the perspective of labor
variables. But this sector has increased in terms of its value added share
since 1977, so including it in the skill-intensive sector would raise the
measured increase in the value-added share of the skill-intensive sector.
This would serve to increase the size of the effects that we estimate, but
we feel that this effect is somewhat misleading. For this reason we have
decided to err on the side of being conservative and not include Real Estate
in our benchmark specification. Because Real Estate and Chemicals and
Chemical Products are similarly ranked, we have also chosen to exclude both
in our benchmark specification.

However, to assess the possibility that our results are influenced by where
we draw the boundary between the two sectors, we have also done our analysis
using two more expansive definitions of the skill-intensive sector, one that
includes Real Estate and Chemical and Chemical Products, and another that
further includes the next two highest ranking sectors, Electrical and
Optical Equipment\ (average rank 7.3) and Public Administration and Defense\
(average rank 8.1). Using either of these more expansive definition does not
affect our main message. Results are included in the Online Appendix.

While our empirical analysis of aggregate time series data is closely
related to that in \citet{BueKab12} there is a key difference. They divided
industries within the service sector into two mutually exclusive groups: a
high-skill intensive group and a low-skill intensive group, and show that
whereas the value added share of the high-skill intensive group rose
substantially between 1950 and 2000, the value-added share of the low-skill
intensive group actually fell over the same time period. In contrast to
them, we split the entire economy into a high-skill intensive group and a
low-skill intensive group, and not just the service sector. While in our
benchmark specification the skill-intensive sector consists exclusively of
service sectors, our more expansive definitions also include some goods
producing sectors. But importantly, to assess how structural change affects
the aggregate demand for skill, one must include the contribution of all
sectors and not just those within services. Another difference from %
\citet{BueKab12} is that we also report cross-sectional micro evidence on
income effects.

\subsection{Skill-Biased Structural Change}

In this subsection we document the phenomenon of skill-biased structural
change, i.e., the systematic increase in the relative size of the
skill-intensive sector that accompanies development.

\begin{figure}[!hbt]
\caption{ Structural Change by Skill Intensity and Economic Development.}
\label{fig_1}
\smallskip \centering
\begin{minipage}{0.85\textwidth}
\includegraphics[width=\linewidth]{figure1.pdf}
\end{minipage}
\end{figure}

Figure 1 shows the relationship between development, as proxied by real GDP
per capita and the rise of the skill-intensive sector.\footnote{%
The analogous figures for the more expansive definition of the
skill-intensive sector are contained in the Online Appendix.} We use two
different measures for the size of the skill-intensive sector: its share of
total labor compensation, and its share of total value added. Labor
compensation is more relevant from the perspective of labor demand, but
value added is the more typical metric for theories of structural change. We
include country-level fixed effects with the US being the excluded country.
The left panel of Figure 1 shows the relationship using labor compensation,
while the right panel shows the relationship using value added. The small
squares show the relationship for countries other than the US, and the
larger circles represent data for the US.

Both panels lead to the same conclusion: the relative size of the
skill-intensive sector increases with log GDP per capita, with highly
significant (at $0.1$ percent levels) semi-elasticities of $0.21$ and $0.15$
respectively.\footnote{%
The $R^2$ values for these regressions are $0.93$ and $0.92$. If we exclude
log GDP per capita and only have fixed effects the $R^2$ values are $0.48$
and $0.49$, indicating the time series variation in GDP per capita accounts
for a large part of the time series variation in the size of the high-skill
sector.} The regression line implies an increase of roughly $30$ percentage
points of labor compensation and $20$ percentage points of value added, as
we move from a GDP per capita of $10,000$ to $40,000$ (in 2005 PPP terms).%
\footnote{%
The fact that the semi-elasticity for compensation is significantly higher
than for value added will be relevant when we compare our findings with
those of \citet{KatMur92} later in the paper.} Moreover, we see that the
relationship found in the US data is quite similar to the overall
relationship. Indeed, the tight relationship suggests that from the
perspective of time series changes, cross-country differences in the details
for funding of education or health, for example, are second order relative
to the income per capita relationship in terms of their effects. (Recall
that we have removed country fixed effects in Figure 1.) In sum, the
tendency for economic activity to move toward skill-intensive industries as
an economy develops is a robust pattern in the cross-country data.

\subsection{Structural Change Mechanisms}

One common explanation for structural change is changes in relative prices %
\citep[see, for example,][]{Bau67,NgaPis07}. Using value added price indices
from the EUKLEMS Database, we examine the correlation between changes in the
relative price of the skill-intensive sector and the changes in its value
added share that accompanies the process of development.\footnote{%
We construct sector-level aggregate indices as chain-weighted Fisher price
indices of the price indices for individual industries. Calculation details
are available in the Online Appendix.} Figure 2 is analogous to Figure 1,
but plots the value added price index of the high skill-intensive sector
relative to the low skill-intensive sector.\footnote{%
The analogous figure for the more expansive definition of the high-skill
sector is contained in the Online Appendix.} We have again taken out country
fixed effects, and have normalized the relative price indices to $100$ in
1995. As before, the larger circles represent the U.S. data. 
\begin{figure}[tbh]
\caption{ Relative Price of Skill-intensive Sector and Economic Development.}
\label{fig_2}
\smallskip \centering
\begin{minipage}{0.85\textwidth}
\includegraphics[width=\linewidth]{figure2.pdf}
\end{minipage}
\end{figure}

Figure 2 reveals a strong positive relationship between the relative price
of the skill-intensive sector and development.\footnote{%
The $R^{2}$ for this regression is $0.76$. If we exclude log GDP per capita
the $R^{2}$ is only $0.10$.} In this case the relationship in the US data is
a bit steeper than in the overall data set, but the strong relationship
exists even abstracting from the US. We conclude that changes in relative
prices are another robust feature of the structural transformation process
involving the movement of activity toward the skill-intensive sector.

A second common explanation for structural change is income effects
associated with non-homothetic preferences \citep[see, for example,][]{KRX01}%
. With this in mind it is of interest to ask whether the output of the
skill-intensive sector is a luxury good, i.e., has an income elasticity that
exceeds one. To pursue this we examine the relationship between the skill
intensity of value-added consumption and income in the Consumer Expenditure
Survey (CEX).\footnote{\citet{Leo15} carries out a closely related exercise
and concludes that higher income and more educated individuals have higher
expenditure shares on final expenditure categories that rely more on
high-skill workers, even when taking intermediate input use into account.}
To the extent that all households face the same prices at a given point in
time and have common preferences (or at least preferences that are not
directly correlated with income), the cross-sectional expenditure patterns
within a country abstract from the relative price relationship in Figure 2
and allow us to focus on the effect of income holding prices constant.

Having constructed household level value added consumption expenditure
shares as noted earlier, we regress this share on household observables,
most importantly income or education, and potentially a host of other
household level controls. Our analysis is similar in spirit to that in %
\citet{AguBil15} with two exceptions.\footnote{%
This exercise is also related to the analysis in \citet{Leo15}.} First, they
do not consider our two-sector skill-intensity aggregation, and second, they
study final expenditure elasticities whereas we consider value added
expenditure elasticities.

Table 1 presents results for regressions of the total share of expenditures
that is derived from the skill-intensive sector. The first column presents
results from an OLS regression on log after tax income and a set of
demographic controls, including age, age squared, dummies for sex, race,
state, urban, and month, and values capturing household composition (number
of boys aged 2-16, number of girls aged 2-16, number of men over 16, number
of women over 16 years, and number of children less than 2 years). The
coefficient on log income in the first column indicates that the
semi-elasticity of the skill-intensive share of value added embodied in
expenditures is $0.030$. The second column replaces log income with the log
of total expenditures, and finds a larger semi-elasticity of $0.050$.%
\footnote{%
The larger coefficient for expenditures may be driven by certain lumpy
expenditures like higher educational expenses and car purchases driving both
up in particular months. We nonetheless report these coefficients for the
sake of completeness.}

\begin{table}[tbph]
\begin{center}
\begin{threeparttable}
\begin{tabular}{lccccc}
\multicolumn{6}{c}{\small Table 1} \\ 
\multicolumn{6}{c}{\small Household Skill-Intensive Expenditure Share vs. Income or Total Expenditures } \\ \hline 
& OLS\phantom{**} & OLS\phantom{**} & IV\phantom{**} & IV\phantom{**} & OLS\phantom{**} \\ 
\hline  
\textbf{Ln Income}          & $0.030^{***}$             & -$\phantom{**}$         & $0.054^{***} $          & -$\phantom{**} $         & -$\phantom{**} $        \\ 
            & $(0.001)\phantom{**}$     & -$\phantom{**}$          & $(0.002)\phantom{**}$   & -$\phantom{**}$          & -$\phantom{**}$          \\ 
\textbf{Ln Expenditures}    & -$\phantom{**}$           & $0.050^{***} $           & -$\phantom{**}$         & $0.081^{***} $           & -$\phantom{**}$          \\ 
            & -$\phantom{**} $          & $(0.002)\phantom{**}$    & -$\phantom{**}  $       & (0.002)\phantom{**}    & -\phantom{**}           \\ 
\textbf{High-skill Head}    & -$\phantom{**} $          & -$\phantom{**} $         & -$\phantom{**}$         & -$\phantom{**} $         & $0.047^{***}  $           \\ 
            & -$\phantom{**} $          & -$\phantom{**} $         & -$\phantom{**} $        & -$\phantom{**}$         & $(0.002)\phantom{**}$    \\ 
$R^{2}$                     & $0.19\phantom{**} $       & $0.22\phantom{**}  $     & $0.12\phantom{**}$      & $0.16\phantom{**}  $     & $0.18\phantom{**} $     \\ 
\textbf{Observations}       & $13,144\phantom{**}$      & $13,210\phantom{**} $    & $13,144\phantom{**}$    & $13,210\phantom{**}$     & $4,056\phantom{**}$ \\ \hline
\end{tabular}

\begin{footnotesize}
\begin{tablenotes}
\item[1] *** indicate significance at the 1 percent level.\\ 
\item[2] Standard errors are in parenthesis. Controls include: age; age squared; dummies for sex, race, state, urban, and
month; number of boys (2-16 year); number of girls (2-16 years); number of
men (over 16 years); number of women (over 16 years); and number of infants
(less than 2 years). High-skill is defined as 16 years of schooling
attained, while low-skill is defined as 12 years attained. Sample includes households with heads aged 25-64 and complete income data.
\end{tablenotes} 
\end{footnotesize} 
\end{threeparttable}
\end{center}
\end{table}

Both income and expenditure are certainly subject to measurement error, and
even if properly measured, income would only proxy for permanent income,
leading to a likely attenuation bias. The third and fourth columns attempt
to alleviate this measurement error by instrumenting for log income or log
expenditures, respectively, using the years of schooling attained by the
head of household. Instrumenting for income in this fashion increases the
coefficient almost two-fold to $0.054$. Likewise, instrumenting for log
total expenditures increases the coefficient by about 60 percent to $0.081$.

The last column uses education as a direct regressor, replacing log income
or log expenditures with a dummy for whether the head of household is high
skill or not. Here high-skill is defined as having exactly 16 years of
education, while low-skill is defined as having exactly 12 years. (The rest
of the households are dropped, leading to the smaller sample size.) The
coefficient indicates that the skill-intensive share of value added embodied
in expenditures is $4.7$ percentage points higher in households with a
high-skill head.

We have examined the robustness of the results in Table 1 along various
dimensions. Table 1 uses \textquotedblleft quarterly\textquotedblright\
expenditures of the household across the three months they are surveyed, but
if we use the monthly data directly, we find nearly identical results.
Dropping demographic controls increases the sample by about twenty percent,
but again the coefficients are essentially unchanged and highly significant.
By defining high-skill as those with at least 16 years of education, and low
skill as those with less than 16 years of education, we expand the sample
somewhat; the raw coefficient is slightly smaller but not dramatically so ($%
0.029$ rather than $0.047$). The coefficient remains highly significant. We
also examined the diary sample, a smaller sample with a survey that focuses
on higher frequency expenditures. In the diary data, we estimate the same
coefficient for expenditures but the coefficients on income ($0.023$ vs. $%
0.030$) and high-skill head ($0.022$ vs. $0.045$) are slightly smaller.%
\footnote{%
Average monthly expenditures in the diary survey are less than ten percent
of the average monthly expenditures in the interview survey.}

Recalling that the aggregate time series data in Figure 1 implied a
coefficient of $0.17$ on log GDP per capita without controlling for changes
in relative prices, the instrumented expenditure coefficient of $0.081$
suggests that a significant part of the aggregate time series effect may be
driven by income effects. We therefore take this as evidence that, in
addition to relative prices, non-homotheticities may also play a role in
accounting for the observed pattern of skill-biased structural change

Lastly, we note an important limitation in directly applying the micro
elasticity as an income effect. Because the CEX captures only out-of-pocket
expenditures, it underestimates the true consumption of certain goods like
health care (a substantial share of which is paid by employers for working
individuals and by the government for those on Medicare), and education (a
substantial share of which is paid by government).\footnote{%
The estimated income semi-elasticity of the share of out-of-pocket insurance
is actually significantly negative in the CEX data although overall
insurance consumption is certainly positive. Similarly, although the
expenditure share income semi-elasticity of higher education is positive, it
is likely understated. Finally, the lack of primary and tertiary
expenditures may actually be overstated in the CEX data because it neglects
public expenditures, but we conjecture that this relationship is small
relative to the higher education relationship.} This caution
notwithstanding, we will use our estimated elasticity of $0.81$ in our
calibration exercise as a way to discipline the relative importance of
income and relative price effects.\footnote{\citet{Bop14} also used micro
data to discipline these effects, though he used a different two-sector
aggregation and also studied final expenditure shares rather than value
added shares. Differently from him we will simply use a reduced form
elasticity to calibrate our model whereas he used micro data to estimate
structural preference parameters.}

\subsection{Summary}

We have documented a robust relationship in the time series data for
advanced economies regarding the systematic movement of activity into the
skill-intensive sector associated with the process of development. We refer
to this process as skill-biased structural change, so as to emphasize both
its connection to the traditional characterization of structural change and
the special role of skill intensity. This relationship is remarkably stable
across advanced economies, thus suggesting that it is explained by some
economic forces that are robustly associated with development, with country
specific tax and financing systems not playing a central role in explaining
the time series changes.

The traditional structural change literature emphasizes the role of both
income and relative price changes as drivers of structural change. We have
presented evidence that both of these effects seem relevant in the context
of skill-biased structural change as well.

\section{Theoretical Framework}

Our analysis emphasizes how intratemporal equilibrium allocations are
affected by changes in the economic environment that operate through changes
in income and relative prices. To capture these interactions in the simplest
possible setting, we adopt a static, closed economy model with labor as the
only factor of production.\footnote{%
We later carry out an exercise to assess how changes in net trade flows by
sector affect our key findings.} Our model is essentially a two-sector
version of a standard structural transformation model extended to allow for
two labor inputs that are distinguished by skill. In this section, we
describe the economy and its equilibrium at a point in time; and derive
analytic expressions that capture the key economic mechanism at work in our
model that connects technical change, structural change, and the skill
premium.

\subsection{Model}

There is a unit measure of households. A fraction $f$ are high-skill, and
the remaining fraction $1-f$ are low-skill. All households have identical
preferences defined over two commodities. In our quantitative analysis these
two commodities will be connected to the low- and high-skill intensive
sectors defined in the previous section. In our benchmark specification all
of the high-skill intensive sectors are services and all goods sectors are
in the low-skill intensive sector. It is notationally convenient to label
the two commodities as goods and services even though what we label as goods
includes some service sectors.

Preferences take the following form:

\begin{equation*}
U(c_{G},c_{S})=\left[ a_{G}c_{G}^{\frac{\varepsilon -1}{\varepsilon }%
}+(1-a_{G})\left( c_{S}+\bar{c}_{S}\right) ^{\frac{\varepsilon -1}{%
\varepsilon }}\right] ^{\frac{\varepsilon }{\varepsilon -1}}
\end{equation*}
where $c_{G}$ and $c_{S}$ are consumption of goods and services, $0<a_{G}<1$%
, $\bar{c}_{S}\geq 0$ and $\varepsilon >0$. Note that if $\bar{c}_{S}>0$,
preferences are non-homothetic and, holding prices constant, the expenditure
share on services will be increasing in income.\footnote{%
This is a simple and common way to create differential income effects across
the two consumption categories. One can also generate non-homothetic demands
in other ways. For example, \citet{HalJon07} generate an income elasticity
for medical spending that exceeds unity through the implied demand for
longevity. \citet{Bop14}, \citet{Swi14} and \citet{CLM15} all consider more
general preferences with the common feature being that income effects
associated with non-homotheticities do not vanish asymptotically. This
property is likely to be relevant when considering a sample with countries
at very different stages of development. Because we focus on a sample of
predominantly rich countries, we have chosen to work with the simpler
preference structure in order to facilitate transparency of the economic
forces at work.} This non-homotheticity is motivated by the cross-sectional
analysis in the previous section. Note that households are assumed to not
value leisure, since our focus will be on the relative prices of labor given
observed supplies.

Each of the two production sectors has a constant returns to scale CES
production function that uses low- and high-skill labor as inputs:

\begin{equation*}
Y_{j}=A_{j}\left[ \alpha _{j}H_{j}^{\frac{\rho -1}{\rho }}+\left( 1-\alpha
_{j}\right) L_{j}^{\frac{\rho -1}{\rho }}\right] ^{\frac{\rho }{\rho -1}%
}j=G,S
\end{equation*}%
where $L_{j}$ and $H_{j}$ are inputs of low- and high-skill labor in sector $%
j$, respectively, $\alpha _{j}$ captures skill-biased technical change in
sector $j$ and $A_{j}$ reflects skill-neutral technical change in sector $j$%
. Our benchmark specification assumes that $\rho $, the elasticity of
substitution between low- and high-skill labor, is the same in both sectors.%
\footnote{%
We consider the effects of cross-sectional variation in this parameter in
the Online Appendix.}

Our representation of technical change merits some discussion. Technical
change in each sector is two-dimensional, and can be represented in many
equivalent ways. Our chosen representation is particularly convenient for
the effects that we will emphasize. Specifically, in settings with an
aggregate production function the skill premium is affected only to the
extent that technical change affects the relative marginal products, and is
independent of what happens to overall output. Our analysis will focus on
the effects associated with the component of technical change that affects
output without affecting relative marginal products. The above
representation is convenient\ relative to common alternatives because
changes in $\alpha $ will have a first-order effect on relative marginal
products but a dampened effect on output due to the opposing effects
embedded in the specification.\footnote{%
More generally, consider a two factor CRS production function $F(H,L)$. One
natural representation of technical change is $F(A_{H}H,A_{L}L)$. In this
case changes in either of the $A_{i}$ generate first order effects on both
output and relative marginal products. Two alternative representations that
partially address this are to write either $A_{H}F(H,\frac{A_{L}}{A_{H}}L)$
or $A_{L}F(\frac{A_{H}}{A_{L}}H,L)$. But in each case there is still a
first-order effect of changes in $A_{H}/A_{L}$ on output. While our
specification dampens the effect of skill-biased technical change on output,
it does not completely eliminate this effect. We have carried out a
sensitivity exercise in which we allocate part of the change in the $A_{j}$
to the skill-biased component, so that both the direct and indirect (i.e.,
general equilibrium) effect of skill-biased technical change on aggregate
output is exactly equal to zero. This has a modest effect on our results and
so is included in the Online Appendix.}

We emphasize that our representation does not imply that we view changes in
the $A_{j}$ and the $\alpha _{j}$ as two independent processes; rather, our
representation is simply a decomposition of the process of technical change
into two components. Any pattern of factor augmenting technical change can
be decomposed into these two pieces.

Before proceeding to analyze the equilibrium for our model we comment on the
significance of abstracting from capital and trade. By excluding capital we
implicitly adopt a somewhat reduced form view of skill-biased technological
change. For example, changes in relative demand for skilled labor due to
capital-skill complementarity and changes in the price of equipment %
\citep[as in][]{KORV00} will show up in our model as skill-biased
technological change. While it is obviously of interest to understand the
underlying mechanics of skill-biased technological change, we believe our
results are strengthened by adopting a more expansive notion of skill-biased
technological change rather than focusing on a particular mechanism.%
\footnote{\citet{AceGue08} emphasize that capital accumulation may also be a
cause of structural change. In our framework these effects will be captured
by changes in the $A_{j}$.}

Although our benchmark analysis abstracts from trade, it implicitly captures
some potential effects of trade. In particular, changing patterns of trade
may affect the composition of production within the low-skill intensive
sector due to specialization. If this involves specialization in higher
skill sectors within our low-skill intensive sector, our analysis will
capture this as skill-biased technical change within the low-skill intensive
sector. That is, part of what we measure as skill-biased structural change
within the low-skill intensive sector may reflect the effects of trade.

A separate issue is that as trade in services has increased over time, it
may also contribute to the changing composition of US production across the
low- and high-skill intensive sectors. In Section 7.2 we carry out an
exercise to assess the importance of this effect.

\subsection{Equilibrium}

We focus on a competitive equilibrium for the above economy. The competitive
equilibrium will feature four markets: two factor markets (low- and
high-skill labor) and two output markets (goods and services), with prices
denoted as $w_{L}$, $w_{H}$, $p_{G}$ and $p_{S}$. We will later normalize
the price of low-skill labor to unity so that the price of high-skill labor
will also represent the skill premium.

The definition of competitive equilibrium for this model is completely
standard, so we move directly to characterizing it. Individuals of skill $%
i=L,H$ solve 
\begin{equation*}
\max_{\left\{ c_{Gi},c_{Si}\right\} }\left[ a_{G}c_{Gi}^{\frac{\varepsilon -1%
}{\varepsilon }}+(1-a_{G})\left( c_{Si}+\bar{c}_{S}\right) ^{\frac{%
\varepsilon -1}{\varepsilon }}\right] ^{\frac{\varepsilon }{\varepsilon -1}}
\end{equation*}%
subject to 
\begin{equation}
p_{G}c_{Gi}+p_{S}c_{Si}=w_{i}.  \label{eq:BC}
\end{equation}

Using the first-order conditions of this problem and normalizing $w_{L\text{ 
}}$to unity, the aggregate expenditure share for services, denoted by $e_{S}$
satisfies: {\small 
\begin{align}
e_{S} & =\frac{p_{S}\left[ (1-f)c_{SL}+fc_{SH}\right] }{1-f+fw_{H}} = \frac{1%
}{\left( \frac{1-a_{G}}{a_{G}}\right) ^{\varepsilon }+\left( \frac{p_{G}}{%
p_{S}}\right) ^{1-\epsilon }}\left[ \left( \frac{1-a_{G}}{a_{G}}\right)
^{\varepsilon }-\frac{p_{S}\bar{c}_{S}\left( \frac{p_{G}}{p_{S}}\right)
^{1-\epsilon }}{1-f+fw_{H}}\right] .  \label{shh}
\end{align}
}{\normalsize This expression illustrates the two forces driving structural
change from the perspective of the household: relative prices and income.
Specifically, if $\epsilon <1$, the expenditure share of services increases
as $p_{G}/p_{S} $ declines, and if $\bar{c}_{S}>0$, the expenditure share of
services increases as income measured in units of services (i.e., $%
(1-f+fw_{H})/p_{S}$) increases. }

{\normalsize The problem of the firm in sector $j=G,S$ is 
\begin{equation*}
\max_{\left\{ H_{j},L_{j}\right\} }\text{ }p_{j}A_{j}\left[ \alpha
_{j}H_{j}^{\frac{\rho -1}{\rho }}+\left( 1-\alpha _{j}\right) L_{j}^{\frac{%
\rho -1}{\rho }}\right] ^{\frac{\rho }{\rho -1}}-w_{H}H_{j}-L_{j}.
\end{equation*}%
Cost minimization plus the requirement that profits be zero in a competitive
equilibrium for a firm with a constant returns to scale production function
imply an equation for the price of sector $j$ output in terms of the skill
premium $w_{H}:$ 
\begin{equation}
\hat{p}_{j}\left( w_{H}\right) =\frac{1}{A_{j}}\left[ \frac{\alpha
_{j}^{\rho }}{w_{H}^{\rho -1}}+(1-\alpha _{j})^{\rho }\right] ^{\frac{1}{%
1-\rho }}.  \label{eq:pi(w)}
\end{equation}%
It follows that finding equilibrium prices can be reduced to a single
dimension: if we know the equilibrium value of $w_{H}$ then all of the
remaining equilibrium prices can be determined. }

{\normalsize Equilibrium requires that all four markets clear: the two
markets for output and the two markets for labor. Here we derive an
expression for the market-clearing condition for high-skilled labor that
contains the single price $w_{H}$. Using $H_{j}/L_{j}=\left( \frac{\alpha
_{j}}{1-\alpha _{j}}\frac{1}{w_{H}}\right) ^{\rho }$, the production
function of sector $j$, and (\ref{eq:pi(w)}), we obtain a sector-specific
demand function for high-skilled labor: 
\begin{equation}
H_{j}=\bigg[\frac{\alpha _{j}\hat{p}_{j}(w_{H})A_{j}}{w_{H}}\bigg]^{\rho }%
\frac{Y_{j}}{A_{j},}  \label{H_i}
\end{equation}%
which, together with equilibrium in the goods market, yields the
market-clearing condition for high-skilled labor solely as a function of $%
w_{H}$: 
\begin{eqnarray}
\left[ \frac{\alpha _{S}\hat{p}_{S}(w_{H})A_{S}}{w_{H}}\right] ^{\rho }\frac{%
f\hat{c}_{SH}\left( w_{H}\right) +(1-f)\hat{c}_{SL}\left( w_{H}\right) }{%
A_{S}} &&  \notag \\
+\left[ \frac{\alpha _{G}\hat{p}_{G}(w_{H})A_{G}}{w_{H}}\right] ^{\rho }%
\frac{f\hat{c}_{GH}\left( w_{H}\right) +(1-f)\hat{c}_{GL}\left( w_{H}\right) 
}{A_{G}} &=&f.  \label{SMC}
\end{eqnarray}%
Here we have used $\hat{c}_{ji}(w_{H})$ to denote the demand for output of
sector $j$ by a household of skill level $i$ when the high-skilled wage is $%
w_{H}$ and prices are given by the functions $\hat{p}_{j}(w_{H})$ defined in
(\ref{eq:pi(w)}). }

\subsection{Structural Change and the Skill Premium}

{\normalsize In this subsection we derive an analytic expression that
summarizes how the sector-specific skill-neutral component of technical
change affects the skill premium. To do this we focus on the special case in
which preferences are homothetic (i.e., $\bar{c}_{S}=0$) and solve for a
linear approximation of the model (i.e., equations (\ref{eq:BC}), (\ref{shh}%
), (\ref{eq:pi(w)}), and (\ref{SMC})). The resulting log-linear expression
is: 
\begin{equation*}
\frac{dw_{H}}{w_{H}}=\frac{1-f+fw_{H}}{(1-f)\tilde{\rho}}\left(
h_{S}-e_{S}\right) \left( 1-\varepsilon \right) \left( \frac{dA_{G}}{A_{G}}-%
\frac{dA_{S}}{A_{S}}\right)
\end{equation*}%
where $h_{j}=H_{j}/f$, $e_{S}=p_{S}\left[ (1-f)c_{SL}+fc_{SH}\right]
/(1-f+fw_{H})$, and $\tilde{\rho}$ is given by: 
\begin{equation*}
\tilde{\rho}=\rho \frac{\left( 1-f+fw_{H}\right) }{(1-f)}\Bigg[ \Big( \left(
1-\theta _{S}\right) h_{S}+\left( 1-\theta _{G}\right) h_{G}\Big) %
+\varepsilon \left( \theta _{S}-\theta _{G}\right) \left( h_{S}-{e}%
_{S}\right) \Bigg] ,
\end{equation*}%
where $\theta _{j}=w_{H}H_{j}/(p_{j}Y_{j})$. }

{\normalsize Several results follow. First, proportional changes in the $%
A_{j}$ have no impact on the skill premium. (Recall that this derivation
assumed homothetic preferences.) Second, changes in the relative value of
the $A_{j}$ will have no effect on the skill premium if $\varepsilon =1$.
(Notably, these results parallel standard results in the structural change
literature regarding conditions under which technical change generates
structural change.) Third, assuming changes in the relative values of the $%
A_{j}$, and that $\varepsilon \not=1$, \ there will be an effect on the
skill premium if and only if there is heterogeneity in skill intensity,
i.e., if and only if the share of high-skill labor in services, $h_{S}$,
differs from the expenditure share of services, $e_{S}$. In particular, if $%
\frac{dA_{G}}{A_{G}}-\frac{dA_{S}}{A_{S}}>0$, and $\varepsilon <1$, then the
skill premium will increase if and only if the service sector is more skill
intensive, i.e., $h_{S}>e_{S}$. }

{\normalsize The effect of changes in the $A_{j}$ on the skill premium are
intimately related to structural change: the change in relative prices in
the log linearized model is proportional to $\frac{dA_{G}}{A_{G}}-\frac{%
dA_{S}}{A_{S}}$ and the extent to which this change in relative prices
affects expenditure shares is dictated by the value of $(1-\varepsilon )$. }

{\normalsize For future purposes it is also of interest to derive an
expression for the effect of a change in the supply of skill on the skill
premium. Our log linearization yields: }

{\normalsize 
\begin{equation}
d\log{w_{H}}=-\left(\frac{1}{\tilde{\rho}}\right)d\log{\left(\frac{f}{1-f}%
\right)}  \label{effectiverho}
\end{equation}%
where $\tilde{\rho}$ is as defined above. }

{\normalsize Equation (\ref{effectiverho}) highlights the extent to which
our two-sector model generalizes the expression for the elasticity of the
skill premium to a change in the supply of skills relative to a one sector
model. In a one sector model this elasticity is completely determined by the
elasticity of substitution in production and equals $-1/\rho $. In our two
sector model, the effective aggregate elasticity of substitution between the
two types of labor is potentially different due to the fact that one can
substitute labor across sectors. This can either amplify or dampen the
effective elasticity of substitution relative to a one sector model.%
\footnote{%
Intuitively, we see that $\tilde{\rho}=\rho $ if we ignore the general
equilibrium impacts of the skill premium on the price of sectoral output and
of the changes in the supply of high-skill labor on the demand for
high-skill labor. That is, if we set $\rho (\theta _{S}h_{S}+\theta
_{G}h_{G})+\varepsilon (\theta _{S}-\theta _{G})(h_{S}-c_{S})=0$ and $%
f(w_{H}-1)=0$, respectively. Of course, $\tilde{\rho}=\rho $ would also hold
if these two general equilibrium effects happen to perfectly offset each
other.} If $\varepsilon =0$, the two-sector elasticity is smaller than the
elasticity of substitution in production, $\tilde{\rho}<\rho $, \ but for $%
\varepsilon $ sufficiently high the reverse holds, i.e., $\tilde{\rho}>\rho $%
. We will use this expression in the next section when we calibrate the
value of $\rho $. }

\section{Calibration}

{\normalsize In this section we calibrate the model of the previous section
so as to be consistent with the salient features of structural change,
growth, and the changes in the skill premium under the assumption that the
driving forces are changes in technology (both skill-biased and
skill-neutral) and changes in the relative supply of skill.\footnote{%
To the extent that factors such as changes in the minimum wage and
unionization affect the skill premium, our analysis will identify them as
changes in skill-biased technical change. This was also the case for the
analysis of Katz and Murphy (1992). Our estimate of the contribution of
skill-biased technical change should be understood as including the effects
of these other factors.} In particular, we will use the above model to
account for observed outcomes at two different points in time, that we
denote as $0$ and $T$ for the initial and terminal periods respectively.
Consistent with the existing literature on technological change and the
skill premium, we do not allow the parameter $\rho $ to change over time. We
also assume that preferences are constant over time. }

{\normalsize Calibrating the model in the initial and terminal period
requires assigning values for $14$ parameters. Nine of these are technology
parameters: four values of the $\alpha _{j}$ (two in each period), four
values of the $A_{j}$ (two in each period), and $\rho $. Three are
preference parameters: $\varepsilon $, $a_{G}$ and $\bar{c}_{S}$. Lastly we
have the value of $f$ at the initial and terminal dates. The two initial
values of the $A_{j}$ represent a choice of units and the initial and final
values of $f$ will be measured directly from the data. We will calibrate the
elasticity parameter $\rho $ in accordance with existing estimates,
appropriately filtered through our model, as summarized by equation (\ref%
{effectiverho}). (We describe this in more detail at the end of the
section.) This leaves nine parameters to be calibrated, six technology
parameters and three preference parameters. }

{\normalsize Our calibration procedure will proceed in two steps. The first
step describes how we determine the six technology parameters independently
of the three preference parameters. Having determined the six technology
parameters we then describe how we determine the three preference
parameters. }

\subsection{Calibrating Technology Parameters}

{\normalsize In this section we show that the six technology parameters can
be determined independently of the three preference parameters if we target
the following eight values from the data: the initial and final values for
factor shares in both sectors, the initial and final value added shares for
the two sectors, the initial and final value of the skill premium, the
change in the relative price of the two sectors, and the overall growth rate
of the economy.\footnote{%
Recall that in Section 2 we reported evidence for changes in both sectoral
value added shares as well as sectoral compensation shares. Although both
displayed the same pattern, the changes in compensation shares were somewhat
larger. Our baseline calibration uses data on value added shares, but in
Section 3 of the Online Appendix we repeat our main exercise using compensation shares instead.
Our main message is robust to this change, and intuitively, implies a larger
role for skill biased structural change.} }

{\normalsize To measure these targets in the data we rely on the World KLEMS
data for the U.S. for the years 1977 and 2005.\footnote{%
We use World KLEMS rather than EUKLEMS in this exercise to facilitate
comparison with the work by \citet{KatMur92}. They base their analysis on
the CPS, and it turns out that the micro data underlying World KLEMS is much
closer to the CPS data than the micro data underlying EUKLEMS. The reason
for the difference is that EUKLEMS makes adjustments so as to make their
data match data from the BEA. We note however, that although the two
datasets provide slightly different answers for the shift-share calculations
of Katz and Murphy, our model based results are effectively unchanged if we
instead use the EUKLEMS data to provide all of our calibration targets. We
continue to use EUKLEMS for cross-country comparisons because this cannot be
done in World KLEMS.} This period is of particular interest, since 1977
effectively marks a local minimum in the skill premium 
\citep[see][for
earlier data]{AceAut11}, after which it secularly increases.\footnote{{We
choose 2005 as our terminal date because this is the last period available
consistently across datasets.}} }

{\normalsize Many of the targets have obvious counterparts in the data and
so require no discussion. But two issues merit some discussion. The first
concerns the fact that because our model does not include investment, we
implicitly assume that output value-added shares reflect consumption
value-added shares. We show in our sensitivity analysis that adjusting the
data as in \citet{HRV13} to compute consumption value-added shares has
virtually no impact on the targets used for calibration. }

{\normalsize The second issue concerns the targets for the labor variables.
As we discuss in more detail later on, our procedure for decomposing
compensation into price and quantity is an important point of departure from
the analysis in \citet{KatMur92}, so we next provide some detail on our
method. }

{\normalsize World KLEMS contains data on labor compensation per hour worked
and average hours worked by week by industry, educational attainment, class,
gender, and age groupings, as well as the number of employed individuals in
each of these groupings.\footnote{%
Until 1992 educational attainment is based on years of schooling and
classified into six categories \textquotedblleft less than high
school\textquotedblright , \textquotedblleft some high
school\textquotedblright , \textquotedblleft high school
graduates\textquotedblright , \textquotedblleft some
college\textquotedblright , \textquotedblleft college
graduates\textquotedblright , and \textquotedblleft more than college
graduates\textquotedblright . After 1992, this classification changes to
highest level achieved, being the categories \textquotedblleft 8th grade or
less\textquotedblright , \textquotedblleft grades 9-12 no
diploma\textquotedblright , \textquotedblleft high school
graduates\textquotedblright , \textquotedblleft some college no degree,
associate degree\textquotedblright , \textquotedblleft
BA,BS\textquotedblright , and \textquotedblleft more than
BA\textquotedblright . Whenever we need to compute consistent time series
using these categories, we perform the adjustment suggested in \citet{Jae97}.
\par
There are two classes of workers (employees and self employed) and eight age
groups (14-15, 16-17, 18-24, 25-34, 35-44, 45-54, 55-64, and 65 and over).
Weekly hours are normalized so that weeks worked per year total 52.}
Consistent with our calculations in Section 2, we combine all workers with
less than college completion into our classification of low-skilled, and all
workers with college completion or more into our classification of
high-skilled to calculate labor income shares by skill at both the aggregate
and sectoral level. We use the same sectoral classification as in Section 2. 
}

{\normalsize Setting targets for the skill premium and the relative supply
of skilled workers requires that we decompose labor payments into price and
quantity components. If all workers within each skill type were identical
then we could simply use total hours as our measure of quantity, but given
the large differences in hourly wage rates among subgroups in each skill
type this seems ill-advised. Instead, we assume that each subgroup within a
skill type offers a different amount of efficiency units per hour of work.%
\footnote{%
The Online Appendix contains more details on our procedure.} We normalize
efficiency units within each skill type by assuming one hour supplied by a
high school-educated prime-aged (i.e., aged 35-44) male is equal to one
efficiency unit of low-skill labor and that one hour supplied by a
college-educated prime-aged (i.e., aged 35-44) male is equal to one
efficiency unit of high-skill labor.\footnote{%
While one could obviously normalize units by choosing other reference
groups, this group seems most natural since its uniformly high rate of
participation over time minimizes issues associated with selection.} With
this choice of units, the skill premium is defined as the ratio of
college-educated to high school-educated prime-aged (i.e., aged 35-44) male
wages. This premium rises from $1.33$ in 1977 to $1.88$ in 2005.\footnote{%
Comparing earnings of full time workers using CPS data, Figure 1 in %
\citet{AceAut11} indicates values of 1.48 and 1.89 for 1977 and 2005
respectively. Our measure indicates a fourteen percentage point greater
increase. This difference basically reflects the fact that Acemoglu and
Autor compute a fixed-weight, composition-adjusted average wage for high
school and college graduates of different experience, race, and gender
groups. If we redo their analysis with CPS data but using only male workers
aged 35-44, we find a 52 percentage point increase in the skill premium,
consistent with our measured increase using World KLEMS data.} Note that our
implicit assumption is that differences in wages between different
demographic groups within a given skill category reflect differences in
efficiency units. This interpretation is consistent with standard practice
in the literature on heterogeneous agent models. }

{\normalsize We infer $f$ using the identity that the ratio of labor
compensation equals the product of the skill premium and the relative
quantity of high- to low-skill labor ($f$ and $1-f$, respectively).\footnote{%
Equivalently, one could compute efficiency units of each skill type by using
relative wages within each skill group to infer efficiency units and
directly aggregating efficiency units.} This procedure implies that
high-skill labor was $21\%$ of total labor supply in 1977 and rose to $32\%$
in 2005. }

{\normalsize Table 2 summarizes the values that will be used to calibrate
the technology parameters. }

\begin{center}
{\normalsize {\small 
\begin{tabular}{cccccccccccc}
\multicolumn{12}{c}{Table 2} \\ 
\multicolumn{12}{c}{Values Used to Calibrate Technology Parameters} \\ \hline
$f_{0}$ & $f_{T}$ & $w_{H0}$ & $w_{HT}$ & $\%\Delta \frac{p_{S}}{p_{G}}$ & $%
\%\Delta Y$ & $\theta _{G0}$ & $\theta _{GT}$ & $\theta _{S0}$ & $\theta
_{ST}$ & $\frac{P_{S0}C_{S0}}{Y_{0}}$ & $\frac{P_{ST}C_{ST}}{Y_{T}}$ \\ 
\hline
$0.21$ & $0.32$ & $1.33$ & $1.88$ & $45.6$ & $80.8$ & $0.18$ & $0.36$ & $0.50
$ & $0.65$ & $0.25$ & $0.39$ \\ \hline
\end{tabular}
} }
\end{center}

{\normalsize We now describe the details of how these values are used to
determine the values of the six technology parameters. We begin with the
determination of the $\alpha _{jt}$. Given a value for $\rho $, the four
values of the $\alpha _{jt}$ are pinned down by sectoral factor income
shares and the skill premium, $w_{Ht}$. To see this, from equations (\ref%
{eq:pi(w)}) and (\ref{H_i}) note that the share of sector $j$ income going
to high-skill labor, $\theta _{jt}=\frac{w_{Ht}H_{jt}}{\hat{p}%
_{j}(w_{Ht})Y_{jt}}$, is 
\begin{equation*}
\theta _{jt}=\frac{\alpha _{jt}^{\rho }}{\alpha _{jt}^{\rho }+(1-\alpha
_{jt})^{\rho }w_{Ht}^{\rho -1}}
\end{equation*}%
Therefore, given $\rho $, the skill premium $w_{Ht}$, and data for $\theta
_{Hjt}$, the value of the $\alpha _{jt}$ are given by: 
\begin{equation*}
\alpha _{jt}=\frac{1}{1+\frac{1}{w_{Ht}^{(\rho -1)/\rho }}\left( \frac{%
1-\theta _{Hjt}}{\theta _{Hjt}}\right) ^{\frac{1}{\rho }}}.
\end{equation*}
}

{\normalsize Next we determine the values of the $A_{jt}$'s. As noted
previously, the two values in period $0$ basically reflect a choice of units
and so can be normalized. We will normalize $A_{S0}$ to equal one, and given
the calibrated values for the $\alpha _{j0}$ and the value of $w_{H0}$, we
choose $A_{G0}$ so as to imply $p_{G0}/p_{S0}=1$. A convenient implication
of this normalization is that $p_{GT}/p_{ST}$ is not only the level of
relative price in period $T$ but is also the change in the relative price
between periods $0$ and $T$.\footnote{%
While our main results will only use information from the initial and final
periods, we note that the procedure described here can be used to uncover
the entire sequence of technology parameters from period $0$ to period $T$.} 
}

{\normalsize As is well known in the literature, with identical Cobb-Douglas
sectoral technologies, relative sectoral prices are simply the inverse of
relative sectoral TFPs, so the change in relative prices would therefore
determine the values of the two $A_{jT}$'s up to a scale factor.\footnote{%
This same relation holds more generally, and in particular would also apply
if the sectoral production functions are CES with identical parameters.}
This precise result does not apply to our setting because of sectoral
heterogeneity in the $\alpha _{jt}$'s. Nonetheless, there is still a close
connection between relative sectoral prices and relative values of the $%
A_{jt}$. In particular, using equation (\ref{eq:pi(w)}) for the two sectors
we have: 
\begin{equation}
\frac{A_{Gt}}{A_{St}}=\frac{p_{St}}{p_{Gt}}\left[ \frac{\frac{\alpha
_{Gt}^{\rho }}{w_{Ht}^{\rho -1}}+\left( 1-\alpha _{Gt}\right) ^{\rho }}{%
\frac{\alpha _{St}^{\rho }}{w_{Ht}^{\rho -1}}+\left( 1-\alpha _{St}\right)
^{\rho }}\right] ^{1/(1-\rho )}.  \label{moment_A_A}
\end{equation}
}

{\normalsize The scale factor influences the overall growth rate of the
economy between periods $0$ and $T$, so we choose this scale factor to
target the aggregate growth rate of output per worker. Note that to compute
aggregate output at a point in time (and thus also the growth rate in
aggregate output) it is necessary to know the sectoral distribution of
output. The relations imposed thus far guarantee that maximum profits will
be zero in each sector, but they do not determine the scale of operation.
Intuitively, the split of activity across sectors at given prices will be
determined by the relative demands of households for the two outputs. Below
we describe how preference parameters are chosen to match the sectoral
distribution of value added at both the initial and final date. At this
stage we simply assume this split is the same as in the data. }

{\normalsize To this point we have identified all of the technology
parameters conditional on a value of $\rho $. We postpone a detailed
discussion of the calibration of $\rho $ until the end of this section, but
note here that for our benchmark analysis we set $\rho =1.53$ and that
filtered through equation (\ref{effectiverho}), our value implies an
effective aggregate elasticity very close to the one used by \citet{KatMur92}%
. Table 3 presents the benchmark calibrated values for the technology
parameters.\newline
%\begin{center}
%{\small 
%\begin{tabular}{ccccccc}
%\multicolumn{6}{c}{Table 2} &  \\ 
%\multicolumn{6}{c}{Values Used to Calibrate Technology Parameters} &  \\ 
%\hline\hline
%$f_{L0}$ & $f_{LT}$ & $w_{H0}$ & $w_{HT}$ & $\%\Delta \frac{p_{S}}{p_{G}}$ & 
%$\%\Delta Y$ &  \\ \hline
%$0.78$ & $0.67$ & $1.41$ & $1.90$ & $62.0$ & $70.0$ &  \\ \hline\hline
%$\theta _{G0}$ & $\theta _{GT}$ & $\theta _{S0}$ & $\theta _{ST}$ & $\frac{%
%C_{S0}}{Y_{0}}$ & $\frac{C_{ST}}{Y_{T}}$ &  \\ \hline
%$0.18$ & $0.34$ & $0.54$ & $0.66$ & $0.29$ & $0.44$ &  \\ \hline
%\end{tabular}
%}
%
}

\begin{center}
{\normalsize {\small 
\begin{tabular}{cccccc}
\multicolumn{6}{c}{Table 3} \\ 
\multicolumn{6}{c}{Calibrated Technology Parameters ($\rho =1.53$)} \\ \hline
$\alpha _{G0}$ & $\alpha _{S0}$ & $\alpha _{GT}$ & $\alpha _{ST}$ & $%
A_{ST}/A_{S0}$ & $A_{GT}/A_{G0}$ \\ \hline
$0.29$ & $0.53$ & $0.46$ & $0.65$ & $1.45$ & $2.35$ \\ \hline
\end{tabular}
} }
\end{center}

{\normalsize We note three features from this table. First, and not
surprising given the way in which we grouped industries into the two
sectors, the weight on low-skill labor is greater in the goods sector than
in the service sector at both dates. Second and more interesting is that in
both sectors technological change has an important component that is
skill-biased. In fact, the level rise in $\alpha $ is greater for the goods
sector than the service sector ($0.17$ versus $0.12$). And third, neutral
technological progress is much greater in the goods sector than in the
service sector. The average annual growth rate of $A_{Gt}$ is $2.99\%$,
while the average annual growth rate of $A_{St}$ is only $1.29\%$ per year. }

\subsection{Calibrating Preference Parameters}

{\normalsize We now turn to the issue of determining values for the three
preference parameters: $a_{G}$, $\bar{c}_{S}$ and $\varepsilon $. While the
previous subsection showed that technological change can be inferred without
specifying any of the preference parameters, we cannot evaluate some of the
counterfactual exercises of interest without knowing how relative demands
for the sectoral outputs are affected by changes in prices. }

{\normalsize The calibration of the $A_{jT}$ used information about sectoral
expenditure shares without guaranteeing that observed expenditure shares
were consistent with household demands given prices. Requiring that the
aggregate expenditure share for goods and services are consistent with the
observed values in the data for the initial and terminal date provides two
restrictions on the three preference parameters. Loosely speaking, given a
value for $a_{G}$, requiring the model to match the initial and final
value-added shares requires that the model match the overall amount of
structural change but does not determine the relative contribution of income
effects and relative price effects. These are in turn dictated by the values
of $\bar{c}_{S}$ and $\varepsilon $. Knowledge about one of these parameters
would allow us to infer the other. }

{\normalsize Earlier in this paper we presented evidence on the effect of
income on the relative expenditure share. We also emphasized that estimates
based on the CEX should be treated with caution given they do not include
government expenditure and that this is an important component of overall
spending on skill-intensive sectors such as health and education. }

{\normalsize Alternatively, the empirical literature has provided estimates
of $\varepsilon $ that correspond to the categories of \textquotedblleft
true\textquotedblright\ goods and \textquotedblleft true\textquotedblright\
services, but not for our definitions of the two sectors that are based
purely on skill intensity. However, given that our goods sector does contain
all of the industries that produce goods, while our service sector does
consist entirely of service sector industries, information about the
elasticity of substitution between \textquotedblleft true\textquotedblright\
goods and \textquotedblleft true\textquotedblright\ services is plausibly
informative about the empirically plausible range of values for $\varepsilon 
$ in our model. Recalling that the objects in our utility function reflect
the value added components of sectoral output, the relevant estimates in the
literature would include \citet{HRV13}, \citet{BueKab09}, and \citet{Swi14}.
All of these studies suggest very low degrees of substitutability.\footnote{%
\citet{CLM15} redo the exercise in \citet{HRV13} for a more general class of
preferences and find an elasticity of substitution that is somewhat higher,
around $0.50$, which is our intermediate case.} Based on these studies we
think a reasonable range of values for $\varepsilon $ is $0$ to $0.50$. }

{\normalsize In light of the above partial information about income and
substitution effects, we proceed as follows. We consider three values of $%
\varepsilon $ from the above range: $0.01$, $0.10$, and $0.50$. For each of
these values for $\varepsilon $, we use equation (\ref{shh}) to determine
values for $a_{G}$ and $\bar{c}_{S}$ by requiring the model to match the
initial and final sectoral value added shares. Given these values we compute
the implied income elasticity of the relative expenditure share for the
skill-intensive sector by comparing the consumption expenditure shares and
incomes of low- and high-skill workers in our model. When $\varepsilon =0.10$%
, the implied income elasticity for the skill-intensive expenditure share is 
$0.085$, which is close to the value in column 4 of Table 1. We choose this
as our benchmark specification. }

{\normalsize Our main results turn out to be quite insensitive to the
relative importance of income and substitution effects in generating the
observed amount of structural change. To allow us to explore this
sensitivity even more fully we will also consider the case of $\varepsilon
=1.00$, which implies that structural change is entirely caused by income
effects, since relative prices have no impact on expenditure shares when $%
\varepsilon =1.00$ if preferences are homothetic. Table 4 shows the
calibrated preference values for each of the values of $\varepsilon $. }

\begin{center}
{\normalsize {\small 
\begin{tabular}{lccc}
\multicolumn{4}{c}{Table 4} \\ 
\multicolumn{4}{c}{Calibrated Preference Parameters} \\ \hline
& $\varepsilon $ & $a_{G}$ & $\bar{c}_{S}$ \\ \hline
Benchmark & $0.10$ & $0.99$ & $0.11$ \\ 
Low $\varepsilon $ & $0.01$ & $1.00$ & $0.09$ \\ 
Intermediate $\varepsilon $ & $0.50$ & $0.49$ & $0.24$ \\ 
High $\varepsilon $ & $1.00$ & $0.26$ & $0.87$ \\ \hline
\end{tabular}
} }
\end{center}

{\normalsize The qualitative patterns in this table are intuitive. Recall
that the calibration procedure implies that in each specification the
changes in income, relative prices and aggregate expenditure shares are the
same. Consider the changes as we move from $\varepsilon =0.10$ to $%
\varepsilon =0.50$. This increases the elasticity of substitution between
the two goods, implying a smaller response in relative expenditure shares.
To compensate for this smaller effect, the impact of income changes on
relative expenditure shares must increase, implying a higher value for $\bar{%
c}_{S}$. The higher value for $\bar{c}_{S}$ will in turn lead to a lower
expenditure share on services in the initial period, thereby requiring a
lower weight, $a_{G}$, on the consumption of goods. }

{\normalsize Consistent with the literature that considers \textquotedblleft
true\textquotedblright\ goods and \textquotedblleft true\textquotedblright\
services, we also find that some income effects are needed to rationalize
the data, as $\bar{c}_{S}$ remains positive even when $\varepsilon =0.01$,
which is effectively the case of Leontief preferences and serves to maximize
the role of relative price effects.\footnote{%
To have better sense of magnitudes, in the benchmark case the value of the
non-homotheticity parameter relative to GDP $p_{S}\bar{c}%
_{S}/(1-f_{H}+f_{H}w_{H})=0.24$ and $0.17$ in the initial and final periods,
respectively.} }

\subsection{Calibrating $\rho $}

{\normalsize We are now in a position to describe our procedure for
calibrating the value of $\rho $. Our procedure follows closely the one
originally adopted by \citet{KatMur92}, and followed by many authors
subsequently. They assumed an aggregate CES production function: }

{\normalsize 
\begin{equation*}
Y_{t}=A_{t}[\alpha _{t}H_{t}^{1-\frac{1}{\rho }}+(1-\alpha _{t})L_{t}^{1-%
\frac{1}{\rho }}]^{\frac{\rho }{\rho -1}}
\end{equation*}%
This specification implies: }

{\normalsize 
\begin{equation*}
\Delta \log \left(\frac{w_{Ht}}{w_{Lt}}\right)=\Delta \log \frac{\alpha _{t}%
}{(1-\alpha _{t})}-\left(\frac{1}{\rho }\right)\Delta \log \left(\frac{H_{t}%
}{L_{t}}\right)
\end{equation*}
}

{\normalsize Their strategy was to assume that technical change proceeded at
a constant rate and to identify $\rho $ from the following regression: }

{\normalsize 
\begin{equation}
\log \left(\frac{w_{Ht}}{w_{Lt}}\right)=a_{1}+a_{2}t+a_{3}\log \left(\frac{%
H_{t}}{L_{t}}\right)+\varepsilon _{t}\text{.}  \label{KMregression}
\end{equation}%
They concluded that $\rho =1.41$. }

{\normalsize We follow this same strategy using our data and time period,
but note two differences in implementation. First, as noted earlier, we use
a different procedure to measure efficiency units of labor. Second,
recalling equation (\ref{effectiverho}), the above regression will identify
the value of $\tilde{\rho}$, which we then use to infer the value of $\rho $.%
\footnote{%
For this step we solve for the implied value of $\rho$ at the initial period
in our sample. The implied value of $\rho$ is roughly the same if we take
the final period instead.} Note that the mapping from $\tilde{\rho}$ to $%
\rho $ depends on the preference parameter $\varepsilon $, so that our
procedure implies a different value of $\rho $ for each profile of
preference parameters. }

{\normalsize Despite the difference in time periods and measurement of
labor, our estimated value of $a_{3}$ in equation (\ref{KMregression}) is
very close to that obtained by \citet{KatMur92}; they obtained a value of $%
-0.709$ while we obtain $-0.708$. For our benchmark specification with $%
\varepsilon =0.10$ the implied value of $\rho $ is $1.53$, which is also
quite close to the value of $1.41$ used by \citet{KatMur92}. For $%
\varepsilon $ in the range of $0$ to $0.50$, the variation in $\rho $ is not
that large, varying from $1.42 $ to $1.55$. When $\varepsilon =1.0$ the
change is more substantial, as the implied value of $\rho $ is $1.18$ for
this case. }

\section{Results}

{\normalsize The procedure described in the previous section implies that
our calibrated model will perfectly account for the observed change in the
skill premium between 1977 and 2005. In this section we use the calibrated
model to decompose changes in the skill premium into parts due to the
exogenous driving forces in the model. Our primary objective is to decompose
the effect of technical change on the skill premium into a piece due to the
skill-biased component of technological change and a second piece due to the
sector-specific skill-neutral component of technological change. }

\subsection{Sources of Change in the Skill Premium}

{\normalsize We begin by decomposing the change in the skill premium between
1977 and 2005 into three pieces. The first piece represents the effect of
changes in the supply of skill (i.e., changes in $f$). Consistent with
equation (\ref{effectiverho}), an increase in the supply of skill holding
technology constant will lead to a decline in the skill premium. The value
of $f$ increases from $0.21$ in 1977 to $0.32$ in 2005. In our benchmark
specification (i.e., $\varepsilon =0.10$) this increase in $f$ holding all
else constant would have led to a reduction of $w_{H}$ from $1.33$ to $0.87$.%
\footnote{%
Because we calibrate $\rho $ so as to hold the effective local elasticity of
substitution constant as $\varepsilon $ varies, this change is very nearly
identical for all of our profiles for preference parameters.} }

{\normalsize In reality (and in our calibrated model) the skill premium
increased from $1.33$ in 1977 to $1.88$ in 2005. As just noted, if the only
change had been a change in the supply of skill, then $w_{H}$ would have
decreased to $0.87$ in 2005. From this we conclude that technical change
generated an increase in the demand for skill that collectively increased
the skill premium from $0.87$ to $1.88$, an increase of $1.01$. Our next
goal is to decompose this change of $1.01$ into one part due to the
skill-biased component of technical change (i.e., changes in the $\alpha
_{j} $) and one part due to the sector-specific skill-neutral change
component of technical change (i.e., changes in the $A_{j}$). In the next
subsection we show that the changes in the $A_{j}$ are intimately related to
structural change. }

{\normalsize A natural way to evaluate each of these effects is to start
from the economy with initial technology settings and $f=f_{T}$, which would
imply $w_{H}=0.87 $, and evaluate the effect on $w_{H}$ of moving to final
values for either the $\alpha _{j}$ or the $A_{j}$. If the model were
linear, these two effects would necessarily add up to the total effect of
changes in technology. However, it turns out that there is a relatively
small positive residual that reflects interactions between the two different
changes in technology. Table 5 presents the results of this exercise. }

\begin{center}
{\normalsize {\small 
\begin{tabular}{llcccc}
\multicolumn{6}{c}{Table 5} \\ 
\multicolumn{6}{c}{Effect of Technological Change on the Skill Premium} \\ 
\multicolumn{6}{c}{US, 1977-2005} \\ \hline
&  & $\varepsilon =0.01$ & $\varepsilon =0.10$ & $\varepsilon =0.50$ & $%
\varepsilon =1.00$ \\ \hline
(i) & Total $\Delta w_{H}$ due to Technology & $1.00$ & $1.01$ & $1.01$ & $%
1.02$ \\ 
(ii) & $\Delta w_{H}$ due purely to the $A_{j}$ & $0.18$ & $0.18$ & $0.17$ & 
$0.15$ \\ 
(iii) & $\Delta w_{H}$ due purely to the $\alpha _{j}$ & $0.77$ & $0.77$ & $%
0.76$ & $0.76$ \\ 
(iv) & $\Delta w_{H}$ due to interaction & $0.05$ & $0.06$ & $0.08$ & $0.09$
\\ \hline
(v) & $\%$ Contribution of the $A_{j}$ & $18.1-23.8$ & $18.0-24.0$ & $%
17.2-24.5$ & $15.3-24.4$ \\ \hline
\end{tabular}
} }
\end{center}

{\normalsize The first row of Table 5 presents the total increase in the
skill premium due to all sources of technical change and represents the
difference between the actual skill premium in 2005 and our model implied
counterfactual for what the skill premium would have been if the only change
had been the supply of skill. As noted earlier, this is effectively the same
for all of the specifications. }

{\normalsize The next three rows in Table 5 present the size of the change
due purely to changes in the $A_{j}$, due purely to changes in the $\alpha
_{j}$, and the interaction term. The final row shows the range for the
percent contributions of the change in the $A_{j}$ as we vary the fraction
of the interaction term allocated to changes in $A_{j}$ from zero to one. }

{\normalsize Focusing on the benchmark calibration ($\varepsilon =0.10$)
case for now, the final row of Table 5 shows that the change in the $A_{j}$
account for between $18$ and $24$ percent of the overall change in the skill
premium due to technological change. Put somewhat differently, according to
our calibrated model, if skill-biased technical change had been the only
force affecting the relative demand for skill then the skill premium would
have increased by only $31-37$ percentage points over the period 1977 to
2005 instead of increasing by $55$ percentage points. }

{\normalsize While our results imply that the skill-biased component of
technical change is the dominant source of changes in the skill premium,
some care should be exercised in interpreting this. As we noted earlier, the
literature has noted that changes due to factors such as minimum wages,
unionization and trade will be included as reflecting skill-biased technical
change. Some estimates claim that as much as half of measured skill-biased
technical change might be due to these other factors. \citep[See, for example][]{DFL96} With this in mind, our results suggest a much less
dominant role for skill-biased technical change. }

{\normalsize Looking at the other columns in Table 5 we see that the basic
message about the relative importance of the change in the $A_{j}$ is fairly
similar across the specifications, with the overall range being $15.3-24.5$
percent. This is reassuring given the lack of definitive values for $%
\varepsilon $ and $\bar{c}_{S}$. However, the table does reveal a tendency
for the pure effect of the $A_{j}$ to decrease as we increase the relative
importance of income effects as a driver of structural change. Specifically,
as }$\varepsilon $ {\normalsize increases from 0.01 to 1.00 the pure effect
of the $A_{j}$ decreases from 18.1 to 15.3. It is of interest to understand
why the effect is relatively small. To be sure, this result partly reflects
our calibration strategy: in all of our specifications the overall change in
income and relative prices is the same as in the data, and in each case the
values of $\varepsilon $ and $\bar{c}_{S}$ are such that we necessarily
account for the same change in expenditure shares as in the data. But as we
detail in Appendix 1, the relatively small effect of changes in $\varepsilon 
$ and $\bar{c}_{S}$ also reflects properties of the data that do not rely on
these details of the calibration procedure. For example, it is relevant that
the change in the }$\alpha _{j}$ {\normalsize are found to have virtually
zero effect on relative prices, but this result reflects the fact that two
opposing effects are roughly offsetting in the data. On the one hand,
because the service sector is more intensive in high skilled labor, a given
change in the }$\alpha _{j}${ \normalsize will have a larger impact on
service sector productivity. But we also find that the change in }$\alpha
_{G}$ {\normalsize exceeds the change in }$\alpha _{S}${\normalsize , and
roughly offsets this first effect.}

{\normalsize We conclude that our message of the importance of changes in
the $A_{j}$ for changes in the skill premium is robust to the relative
importance of income and substitution effects in generating structural
change, and even our estimates are fairly robust; that is, conditional on our model being calibrated so as to generate
the amount of structural change that we see in the data, it is less important to determine the mix of income and substitution effects that leads to this change. }

\subsection{Sources of Structural Change}

{\normalsize In the introduction we stressed the fact that aggregate
production function analyses abstract from compositional changes, and that a
key objective of our analysis was to assess the quantitative importance of
the compositional changes that are associated with the process of structural
transformation during development. The previous calculations decomposed the
overall changes in the skill premium due to technology into parts due to the 
$\alpha _{j}$ and the $A_{j}$. In order to make the connection between this
decomposition and compositional changes it is necessary to examine the
connection between structural change and the components of technical change. 
}

{\normalsize To do this we start by assessing the amount of structural
change that would have occurred had there not been any change in the $A_{j}$%
. Results are shown in Table 6. }

\begin{center}
{\normalsize {\small 
\begin{tabular}{lcccc}
\multicolumn{5}{c}{Table 6} \\ 
\multicolumn{5}{c}{Value Added Share of the Skill-Intensive Sector} \\ 
\multicolumn{5}{c}{US, 1977-2005} \\ \hline
& $\varepsilon =0.01$ & $\varepsilon =0.10$ & $\varepsilon =0.50$ & $%
\varepsilon =1.00$ \\ \hline
Model 1977 & $0.25$ & $0.25$ & $0.25$ & $0.25$ \\ 
Model 2005 & $0.39$ & $0.39$ & $0.39$ & $0.39$ \\ 
Model 2005 with fixed $A_{j}$ & $0.21$ & $0.21$ & $0.22$ & $0.24$ \\ \hline
\end{tabular}
} }
\end{center}

{\normalsize The first two rows of the table remind us that the
skill-intensive sector grew significantly between 1977 and 2005, increasing
its share of value added from $25$ percent to $39$ percent. Recall that our
calibrated model perfectly replicates the change in the data. The third row
shows what would have happened if there had not been any changes in the $%
A_{j}$. Significantly, this would have led to a decline in the value-added
share for the skill-intensive sector. Given that (more than) all of the
observed structural change is due to changes in the $A_{j}$, we think it is
appropriate to identify the channel through which the $A_{j}$ affect the
skill premium as reflecting compositional change. }

{\normalsize It is significant that the pure effect of the change in the $%
A_{j}$ on the value added share of the skill-intensive sector is actually
greater than the overall observed change: $0.18$ versus $0.14$. That is, the
amount of structural change generated by the changes in the $A_{j}$ is more
than twenty percent larger than the amount of structural change observed in
the data. It follows that the amount of observed structural change in the
data is not a good estimate of the amount of structural change induced by
the change in the $A_{j}$. The significance of this will be highlighted in
the next section when we contrast our model-based findings with those found
by \citet{KatMur92} using shift-share methods. }

{\normalsize The third row of Table 6 reported the combined effect of the
changes in $f$\ and the $\alpha _{j}$\ on the value-added share of the
skill-intensive sector. It is also of interest to assess the role of each of
these changes separately. It turns out that the changes in the $\alpha _{j}$
and $f$ contribute equally to this modest decline. If we change $f$\ from
its 1977 value to its 2005 value but holding technology constant, we see a
decline in the value added share of the skill-intensive sector to $0.23$.
The reason for this decrease is that the increase in the supply of skill
lowers the skill premium, thereby reducing income and lowering the relative
price of the skill-intensive sector which is more intensive in the use of
skilled labor. Both of these effects serve to shift expenditure away from
the skill-intensive sector. It is interesting to note that an increase in
the supply of high-skill labor does not by itself lead to an expansion of
the sector that is more intensive in its use of high-skill labor. }

{\normalsize The changes in the $\alpha _{j}$ also produce a modest decline
in the value-added share of the skill-intensive sector. This reflects the
net effect of several opposing effects. Because this sector uses high-skill
labor more intensively, a uniform increase in $\alpha $ would have a larger
productivity effect on it, thereby lowering its relative price and shifting
expenditure to the goods sector. But, as noted earlier, the increase in $%
\alpha _{G}$\ is somewhat larger than the increase in $\alpha _{S}$. The
increase in the $\alpha _{j}$ also lead to an increase in the skill premium,
which also tends to increase the relative price of the skill-intensive
sector. }

\section{Comparison With the Literature}

{\normalsize In the previous section we argued that changes in the
composition of demand driven by technical change have played a significant
role in the overall increase in the demand for skill. This finding stands in
sharp contrast to previous findings in the literature, specifically those in %
\citet{KatMur92} and \citet{Leo15}. In this section, we examine the reasons
behind these different conclusions. }

\subsection{Comparison With Katz and Murphy (1992)}

{\normalsize We begin by examining how our results compare with those of %
\citet{KatMur92} (hereafter KM). They employ a shift-share method to assess
the contribution of changes in sectoral composition to the overall increase
in the demand for skill. Specifically, their \textquotedblleft Between
Industry Demand Shift\textquotedblright\ for group $k$ measured relative to
base year employment of group $k$ in efficiency units $\Delta X_{k}^{d}$ is
given by 
\begin{equation*}
\Delta X_{k}^{d}=\frac{\Delta D_{k}}{E_{k}}=\sum_{j}\left( \frac{E_{j,k}}{%
E_{k}}\right) \left( \frac{\Delta E_{j}}{E_{j}}\right) =\frac{\sum_{j}\phi
_{j,k}\Delta E_{j}}{E_{k}},
\end{equation*}%
where $E_{k}$ is group $k$'s employment measured in efficiency units and $%
\phi_{j,k}=\left( E_{j,k}/E_{j}\right) $ is group $k$'s share of total
employment in efficiency units in sector $j$ in the base year. The implied
change in the relative demand for skill associated with changes in the
sectoral distribution of employment can in turn be used to infer the implied
change in the skill premium by using their estimate of the elasticity of
substitution between low and high-skilled labor. They conclude that changes
in sectoral composition accounted for only $10.6\%$ of the increase in the
skill premium between 1979 and 1987. }

{\normalsize There are many differences in details between their study and
ours: the data sources are different (CPS vs World KLEMS), the measure of
payments to workers are different (weekly earnings for full time workers
versus compensation per hour), the time periods are different (1963-1987 vs
1977-2005), and the level of aggregation is different (50 sectors vs 2
sectors). In the appendix we report a series of detailed calculations to
show that none of these differences is of first-order significance in
explaining the very different results. In particular, when we redo the
analysis of KM using our data, our time period, and our level of
aggregation, we find that changes in sectoral composition account for only $%
10.3\%$ of the increase in the skill premium. This result is shown in row
(i) of Table 7. }

{\normalsize A less apparent difference is that our analysis targeted
changes in sectoral composition based on changes in value added shares,
whereas the KM procedure measures changes in sectoral composition based on
labor compensation shares. This is significant because the change in
sectoral compensation shares is greater than the change in sectoral
value-added shares.\footnote{%
We noted this feature of the data in Section 2. For the US, the value added
share of the high-skill sector increased from $0.25$ to $0.39$ between 1977
and 2005 whereas the compensation share of the high-skill sector increased
from 0.27 to 0.47.} }

{\normalsize To make the KM numbers directly comparable to ours we redo the
KM analysis but using changes in value-added shares. This turns out to have
a significant quantitative impact. In particular, row (ii) of Table 7 shows
that redoing the KM shift share calculation with value-added shares as
sectoral weights reduces their estimated contribution of compositional
changes by almost five percentage points, to $5.7\%$. It is this value that
should be compared with our estimated range of $18-24\%$.\footnote{%
Alternatively, we could have redone our benchmark calibration exercise to
target the change in compensation shares rather than the change in value
added shares. If we do this we find a range of $26.4-35.9\%$ for the
contribution of changes in the $A_{j}$ to the change in the skill premium.
In both cases our model based approach implies an effect that is between $%
2.5 $ and $3$ times larger than the corresponding estimate based on the KM
shift share calculation. Details of this alternative exercise are included
in the Online Appendix.} }

{\normalsize In what follows we show that there are two key differences that
account for the fact that our estimate is between $2.5$ and $3$ times larger
than theirs. The first key difference is the method for measuring efficiency
units of labor. The second key difference is that our results rely on model
based simulation rather than shift share calculations. We discuss each in
turn. }

\begin{center}
{\normalsize 
\begin{tabular}{cccccccc}
\multicolumn{8}{c}{\small Table 7} \\ 
\multicolumn{8}{c}{\small Comparison With Katz-Murphy (1992)} \\ \hline
& {\small Years} & {\small Data} & {\small Efficiency Units} & {\small %
\#Sectors} & {\small Method} & {\small Weights} & {\small Contribution } \\ 
\hline
(i) & {\small 1977-05} & {\small WK} & {\small KM} & ${\small 2}$ & {\small %
Shift-Share} & {\small Wages} & ${\small 10.3\%}$ \\ 
{\small (ii)} & {\small 1977-05} & {\small WK} & {\small KM} & ${\small 2}$
& {\small Shift-share} & {\small VA} & ${\small 5.7\%}$ \\ 
{\small (iii)} & {\small 1977-05} & {\small WK} & {\small BKRV} & ${\small 2}
$ & {\small Shift-share} & {\small VA} & ${\small 12.9\%}$ \\ 
{\small (iv)} & {\small 1977-05} & {\small WK} & {\small BKRV} & ${\small 2}$
& {\small Model-based} & {\small VA} & ${\small 18.0-24.0\%}$ \\ \hline
\end{tabular}%
\medskip \vspace{-7.5mm} \bigskip }
\end{center}

{\normalsize As noted earlier, within each skill category, we use relative
compensation to measure relative efficiency units supplied by individuals
and supplied to sectors. Importantly, we allow for efficiency units to vary
across workers within a given education/age/gender cell, since this is how
we account for compensation differences within a given cell. In contrast, KM
assume that all workers within a given cell supply the same number of
efficiency units and measure relative quantities of skilled and less skilled
labor without using individual- or sector-specific data on compensation.%
\footnote{%
The Online Appendix provides a framework for thinking about the issue of
measuring labor services and details the differences between our procedure
and that of \citet{KatMur92}.} As shown by Row (iii) of Table 7, this turns
out to have very significant implications, increasing the estimate based on
the KM shift share methodology from $5.7\%$ to $12.9\%$.\footnote{%
Because our calibrated model replicates all of the values in the data that
go into this calculation it follows that this value also reflects what the
KM procedure would infer from our model generated data.} }

{\normalsize We now turn to the second key difference: our use of a
model-based procedure. Row (iv) of Table 7 shows the significance of moving
from KM's shift-share analysis to our fully solved general equilibrium
evaluation of exogenous shifts in technology parameters. KM acknowledge that
their method might underestimate the underlying contribution of demand
shifts if other factors, e.g., the rise of the skill premium due to
skill-biased technical change, served to dampen the reallocation to
skill-intensive sectors.\footnote{%
Bound and Johnson adjust for the increase in the relative supply of
high-skilled labor without accounting for the fact that the relative wage
nevertheless rose. This appears to account for their much lower estimate
than KM.} But they are unable to quantify the extent to which they
underestimate the effect. Our model-based method enables us to actually
quantify this bias. }

{\normalsize Importantly, because we derive endogenous changes in
composition as the result of exogenous changes in model primitives, we can
map changes in primitives to changes in composition and changes in the skill
premium, rather than trying to map changes in composition into changes in
the skill premium. To understand the reason that this matters, note that the
shift-share calculation uses observed changes in composition to evaluate the
effect of compositional changes. But as we showed at the end of the previous
section, the change in composition that is associated with changes in the $%
A_{j}$ was significantly larger than the observed change in composition.
This implies that the shift share calculation will necessarily underestimate
the effect of the change in the $A_{j}$. As a final remark, we note that our
analysis uses a global solution of the model, whereas shift share
calculations are inherently based on local approximation. }

{\normalsize In summary, while there are many small variations, there are
two important factors that explain why we find a substantially larger role
for skill-biased structural change in accounting for increases in the skill
premium relative to what the earlier literature attributed to industrial
composition. The first is that we use wage data to control for unobservable
differences among workers within a cell. This implies a larger increase in
the demand for efficiency units by the skill intensive sector, thereby
increasing the potential impact of compositional changes on the relative
demand for skill. The second is that our structural approach allows us to
precisely disentangle the role of different driving forces by solving an
explicit model-based, globally-solved counterfactual associated with changes
in exogenous technology parameters. Each of these factors plays a key role.
Measurement differences alone account for a difference of around $7$
percentage points, and the use of a model based procedure implies a
difference of between $5$ and $11$ percentage points. }

\subsection{Comparison With Leonardi (2015)}

{\normalsize \citet{Leo15} also asks if changes in composition might be an
important mechanism through which some changes in economic primitives lead
to changes in the skill premium. Differently than us, he finds that these
effects are relatively small. In particular, his exercise finds that the
channel of compositional shifts explains approximately 6.5\% of the relative
demand shift in the US between 1984 and 2002 (see Table 6 in Leonardi,
2015). In this section we discuss the reasons for the apparently different
findings and show that there is no inconsistency. }

{\normalsize As a first step it is important to summarize the calculations
in \citet{Leo15}. He specifies a two-sector model very similar to ours. One
difference is that in his model, high-skill workers have a relatively higher
expenditure share for the output of the skill-intensive sector for two
reasons. First, as in our framework, preferences are non-homothetic and the
income elasticity for this sector's output is greater than one. Second, he
also allows for preferences to differ across low and high-skill workers, and
in particular, allows high-skill workers to place greater value on the
output of the skill-intensive sector. His calculations are based on a log
linear approximation of the demand system generated by this model. }

{\normalsize His result comes from the following calculations. First, he
calculates the counterfactual percentage change in the skill premium induced
by a pure increase in the relative supply of skilled labor. Second, he
considers an alternative version of the model in which preferences of high
and low-skill workers are identical and homothetic, and then repeats the
previous calculation, i.e., calculates the counterfactual percentage change
in the skill premium that would have occurred from a pure increase in the
relative supply of skilled workers. His estimate for the effect of demand
shifts is calculated by taking the difference in the two percentage changes
just calculated. Comparing this to the total percentage change in the skill
premium he arrives at 6.5\% for the contribution of demand shifts to changes
in the skill premium. }

{\normalsize Both our paper and \citet{Leo15} present model based
calculations about the effect of changes in model primitives on the skill
premium that manifest themselves via changes in sector composition.
Moreover, the two analyses employ very similar models, with the lone
difference being that \citet{Leo15} allows high-skilled workers to have
different preferences. But importantly, the two papers focus on different
changes in fundamentals. Whereas Leonardi's calculation isolates
compositional effects that result from a change in the relative supply of
skills, we isolate compositional effects that result from the
sector-specific skill-neutral component of technical change. Because the two
exercises isolate the effects of different shocks, the different results do
not reflect any inconsistency. }

{\normalsize To pursue this further we can use our model to carry out the
same calculation as Leonardi. The answer that we obtain varies depending
upon the profile of preference parameters that we use, and ranges from $0.3\%
$ to $5.6\%$, with the $5.6\%$ value coming from the extreme case in which $%
\varepsilon =1.00$ and income effects are maximized. Keeping in mind that
the two analyses differ in terms of various details (slight differences in
each of time period considered, model specification and calibration
procedure), we view this result as confirming that there is no inconsistency
across the two studies. We conclude that the differing conclusions are due
to the fact that the two papers document the effects of different changes in
fundamentals. Indeed, Leonardi's and our exercises are complementary in the
sense that the contribution obtained by the counterfactual proposed by
Leonardi must be added to our numbers to obtain the total effect that is
mediated via sectoral composition.\footnote{
Because we do not explicitly target the moments highlighted by
\citet{Leo15}, one might interpret this comparison as a model validation
exercise.} } 

\section{Sensitivity Exercises}

{\normalsize In this section we report on four sensitivity exercises. In the
first subsection we discuss the issue that our model does not contain
investment and how controlling for this would affect our results. In the
second subsection we discuss how allowing for trade would affect our
findings. In the third section we discuss the implications of the
possibility that observed changes in relative prices are biased upward due
to mismeasured output. And in the fourth section we consider how our results
are affected by allowing for a simple extension with endogenous skill
supply. }

\subsection{Consumption Value Added vs Investment Value
Added}

{\normalsize Our analysis emphasizes the consequences of the systematic
changes in the composition of output that accompany development for the
overall demand for skill. Our model abstracts from investment and so
implicitly focuses on systematic changes in the composition of consumption
that accompany development. This raises the issue of whether the systematic
changes in consumption value-added shares mimic the systematic changes in
output value-added shares. Here we address this question and show that the
two are very similar. }

{\normalsize \citet{HRV13} carried out a similar exercise but for the
traditional sectoral classification of agriculture, goods and services. They
describe the procedure in detail in the online appendix to their paper. We
follow their methodology but using our sectoral categories and so refer the
reader to their paper for details. }

{\normalsize Implementing the procedure in \citet{HRV13} uses the Historical
I-O Tables produced by the BEA. Our analysis focuses on the period
1977-2005, and within this time period the I-O Tables are available at five
year intervals from 1977 to 1997 and annually thereafter. }

{\normalsize Before presenting the results we note two points. First, World
KLEMS uses NAICS codes to define sectors, whereas the BEA uses SIC codes, so
the sectoral comparisons are not perfectly matched. Second, the BEA and
World KLEMS data sets use somewhat different measurement methodologies and
as a result the levels of some variables will vary across the two. We have
used the World KLEMS data for our output measures in order to have
consistency with the labor measures that we use. But in view of these two
issues, we are most interested in comparing the implications for the change
in the value-added share of the skill-intensive sector. }

{\normalsize When we carry out this exercise we find that the consumption
value-added share for the skill-intensive service sector increases from $%
21\% $ in 1977 to $36\%$ in 2005. Our calibration exercise used output
value-added shares from World KLEMS, and based on this data, we found that
the output value added share of our service sector increased from $25\%$ in
1977 to $39\%$ in 2005. }

{\normalsize While there are level differences between the two measurements,
the key point for our purposes is that the increase in the share of the
skill-intensive services sector is virtually identical between the two: $%
13.8\%$ for the output value added share in our calibration exercise, versus 
$15.4\%$ for the consumption value added share using the data from the BEA
and the method of \citet{HRV13}. We conclude that purging the data of
investment is not an important concern.\footnote{%
In fact, if we redo our calibration exercise using the consumption valued
added shares from this calculation instead of the output value added shares
that we originally used we find modestly larger effects. We now find that
the contribution of the $A_{j}$ is in the range of 19-26\%.} }

{\normalsize This finding is perhaps not too surprising given the results in %
\citet{HRV20}. They show that similar structural change has happened within
both the consumption sector and the investment sector. In view of this it is
not that surprising that the amount of structural change in consumption is
similar to the amount of structural change found in total output. }

\subsection{Allowing for Trade}

{\normalsize Our benchmark analysis considers a closed economy and so
abstracted from changes in trade as a potential driving force. As we noted
earlier, to the extent that much of trade takes place within the goods
producing sector, it is possible that some of the skill-biased technological
change that we infer reflects changes in composition within our low-skill
intensive sector due to changes in specialization associated with trade.
This alone would not affect our estimate of the contribution of the $A_{j}$
to the overall change in the skill premium, though by diminishing the
contribution of skill-biased technical change it would increase their
relative importance. }

{\normalsize More generally, lower trade costs can lead to greater
specialization and hence higher productivity, so part of the productivity
increases that we measure may result from trade. Our procedure aims to
assess the contribution of productivity increases to the skill premium, but
does not seek to understand the underlying source of the productivity
increase. While we think it is of interest to assess the role of trade as a
source of productivity growth, this issue is separate from the one we
address. We refer the reader to the paper by \citet{CraSot19} for an
analysis of the effects of lower trade costs on the skill premium in a
framework similar to ours. }

{\normalsize But not all trade takes place within the goods sector and the
share of trade accounted for by trade in services is increasing over time.
It is therefore possible that changes in trade patterns may also contribute
to changes in the relative size of the skill-intensive sector. In this
subsection we carry out a simple exercise to assess the potential magnitude
of this effect. In particular, we will take sectoral net trade flows as
given and solve for the equilibrium of our model given these flows. }

{\normalsize Net sectoral trade flows create a wedge between production and
consumption in each sector. If net exports from the skill-intensive sector
are increasing over time, this would imply a decrease in consumption of the
skill-intensive sector output holding labor allocations constant. Hence,
this would create an incentive to increase the share of labor allocated to
the skill-intensive sector in order to increase consumption from that
sector. Similarly, if the imports of goods are increasing over time, then
this would increase the relative consumption of low-skill intensive goods
holding labor allocations fixed, and again create an incentive to reallocate
labor to the skill-intensive sector. It follows that part of the movement of
resources to the skill-intensive sector could be the result of changes in
trade and not necessarily technology. }

{\normalsize To estimate net trade flows for our two sector breakdown we do
the following. From the Balance of Payments Accounts we obtain data on net
trade flows for the \textquotedblleft true\textquotedblright\ goods sector
and the \textquotedblleft true\textquotedblright\ services sector in the US
economy for the full sample, 1977-2005. Over these years, the US ran a trade
deficit in trade in \textquotedblleft true\textquotedblright\ goods, and the
deficit increased from around 1.4 percent of GDP to around 6 percent of GDP.
Over the same time period the US ran a small trade surplus in
\textquotedblleft true\textquotedblright\ services, increasing from around
0.2 percent of GDP to around 0.5 percent of GDP. This trade surplus in
services is to first approximation a trade surplus in skill-intensive
services, as there is a small and relatively constant trade deficit in
low-skill services, so that the overall change in the trade deficit in what
we label the low-skill intensive sector (consisting of both goods and
low-skill services) is to first approximation the same as the trade deficit
coming purely from goods.\footnote{%
Our net export figure for the goods sector is based on total value and is
likely an overestimate of the deficit measured in terms of value added. For
this reason we think our estimates for the effect of trade are likely an
upper bound.} To evaluate these assumptions we can use disaggregated trade
flows in services that are available from the BEA for a subset (1999-2005)
of our simulation years. Splitting these flows into low- and high-skill
intensive services components using our previous definitions and aggregate
net trade flows to correspond to our model defined sectors, we show that our
crude assumption, needed for the longer period, is a very good approximation
over these years. }

{\normalsize Taking net sectoral trade flows as given we implement the same
calibration procedure as before and carry out the same counterfactuals to
decompose the effects of technology. Intuitively, if net exports of the
skill-intensive service sector are increasing over time, our calibration
procedure would imply a lower value of $\bar{c}_{s}$, since the needed
income effect from changes in technology would be reduced. Accordingly, the
implied amount of skill-biased structural change would also be reduced. }

{\normalsize The key message that results is that incorporating changes in
trade has a relatively small effect on our results. In the interest of space
we only report results for the benchmark case of $\varepsilon =.10$. Whereas
our earlier results implied that changes in the $A_{jt}$'s accounted for
between $18$ and $24$ percent of the overall change in the skill premium due
to technical change, we now find that the range is between $16$ and $21$
percent. While the changing net sectoral trade balance does account for some
of the movement of resources into the skill-intensive service sector, we
find this effect to be relatively small. }

\subsection{Mismeasurement of Relative Price Changes}

{\normalsize Here we consider the extent to which mismeasurement of relative
prices might influence our results. Our quantitative analysis utilized
information about changes in the relative price of the skill-intensive
sector. Between 1977 and 2005 this relative price increased by more than
forty-five percent. One possible concern is that price inflation in the
skill-intensive sector might be upward biased because of the failure to
properly account for quality improvements. }

{\normalsize Here we report the results of a simple exercise to assess the
extent to which our conclusions are affected by this possibility. In
particular, consider the case in which the true increase in the relative
price of the skill-intensive sector was only half as much as indicated by
the official data. This means that real value added in this sector increased
by roughly $30\%$ more than indicated by the official data, and aggregate
GDP grew by roughly 7 additional percentage points. Note that this
adjustment has no impact on the increase in the value added share of the
skill-intensive sector. }

{\normalsize We set $\rho =1.53$ and $\varepsilon =0.10$ and carry out the
same calibration procedure as previously. Not surprisingly, given that we
are holding $\varepsilon $ fixed and decreasing the role of relative price
changes, the calibration procedure yields a larger value for $\bar{c}_{S}$,
indicating a larger role for nonhomotheticities. However, we find that the
contribution of the skill-neutral component of technical change is virtually
identical to what we found in our benchmark calculation. So while
mismeasurement of relative price changes has implications for relative
magnitudes of preference parameters, it has virtually no effect on our
assessment of the role of demand factors. This follows naturally with the
result that the different channels are less relevant to our quantitative
impact as the overall amount of structural change. }

\subsection{Endogenizing the Supply of Skills}

{\normalsize In the benchmark analysis we take the observed changes in the
supply of skilled labor as an independent exogenous driving force. But if
changes in the relative supply of skill are driven by changes in the skill
premium, this specification may be inappropriate. Here we consider a simple
extension in which changes in supply are completely due to changes in the
skill premium to assess how this affects our results. }

{\normalsize In particular, we assume a simple reduced form relationship
between the supply of skill and the skill premium, $f_{h}=\bar{f}%
_{h}w_{H}^{\zeta }$. This simple relationship can be interpreted as a steady
state supply function. We calibrate the parameters $\bar{f}_{h}=0.15$ and $%
\zeta =1.25$ so that we match the supply of skills in 1977 and 2005, when
using the benchmark values for the preference and technology elasticities $%
\epsilon =0.1$ and $\rho =1.53$, respectively. We obtain a similar
contribution for the changes in the $A_{jt}$'s in the range of $21-23\%$. }

{\normalsize This result is perhaps not surprising. If changes in the supply
of skill are completely driven by changes in the skill premium then changes
in the supply of skill will have a similar decomposition and so will not
affect the relative contribution of the different components of technology. }

\section{Cross-Country Analysis}

{\normalsize In this section we extend our analysis to the ten other OECD
countries for which the available data exists and that we studied in Section
2.\footnote{%
Cross-country data is only available in EUKLEMS, so all of the results in
this section use this data set. In particular, the results in this section
for the US are based on EUKLEMS rather than World KLEMS, which explains why
there are small differences from the earlier results. But as we emphasized
earlier, the differences are quite minor.} The changes experienced by these
countries differ quite significantly, both with regard to the change in the
skill premium as well as the change in the relative supply of skilled labor.
We view the results presented here as a simple first pass at extending the
analysis to other countries. }

{\normalsize In the interests of space we set $\varepsilon =0.10$ for all
countries and also set $\rho =1.53$ as in our benchmark. We take as given
the changes in the relative supply of skill in each country and infer
country-specific processes for technical change using the same procedure
described earlier. We choose country-specific values for $a_{G}$ and $\bar{c}%
_{S}$ to guarantee that the model generates the amount of structural change
found in the data.\footnote{%
In an earlier working paper version of this paper (\citet{BueKabRog15}) we
showed that the \ implied series for technical change were quite similar
across countries, which we think is reassuring.} }

{\normalsize Using the country-specific calibrated models we carry out a
decomposition exercise for each country corresponding to the results that we
previously showed in Table 5. Results are in Table 8. }

\begin{center}
{\normalsize {\small 
\begin{tabular}{lccc}
\multicolumn{4}{c}{Table 8} \\ 
\multicolumn{4}{c}{Decomposition of $\Delta w_{H}$ Due to Technology (\%)}
\\ \hline
\  & $\Delta A_{j}$ Only & $\Delta \alpha _{j}$ Only & Interaction \\ \hline
Australia & $3.9$ & $86.0$ & $10.1$ \\ 
Austria & $21.1$ & $71.8$ & $7.1$ \\ 
Belgium & $14.8$ & $84.3$ & $0.9$ \\ 
Denmark & $10.0$ & $85.7$ & $4.3$ \\ 
Spain & $20.7$ & $75.6$ & $3.7$ \\ 
Germany & $21.8$ & $76.9$ & $1.3$ \\ 
Italy & $21.3$ & $53.1$ & $25.6$ \\ 
Japan & $11.4$ & $83.3$ & $5.3$ \\ 
Netherlands & $19.6$ & $79.4$ & $1.0$ \\ 
UK & $6.1$ & $69.3$ & $24.6$ \\ 
US & $19.7$ & $74.3$ & $6.0$ \\ \hline
Median & $19.6$ & $76.9$ & $3.5$ \\ \hline
\end{tabular}
} }
\end{center}

{\normalsize The contribution of changes in the $A_{j}$ alone varies
significantly, from a low of $3.9\%$ in Australia to a high of $21.8\%$ in
Germany. But there is also considerable variation in the size of the
interaction term across countries. If we allocate all of the interaction
term to changes in the $A_{j}$ then the range varies from a low of $14.0\%$
(Australia and Denmark) to a high of $46.9\%$ (Italy). }

{\normalsize The key message from this brief examination of other countries
is that the process of skill-biased structural change seems to play a
significant role in many countries. }

\section{Conclusion}

{\normalsize Using a broad panel of advanced economies, we have documented a
systematic tendency for development to be associated with a shift in value
added to skill-intensive sectors. It follows that development is associated
with an increase in the relative demand for high-skill workers. We coined
the term skill-biased structural change to describe this process. We have
built a simple two-sector model of structural transformation and calibrated
it to US data over the period 1977 to 2005 in order to assess the
quantitative importance of this mechanism for understanding the large
increase in the skill premium during this period. We find that technical
change overall increased the skill premium by roughly 100 percentage points,
and that between $18$ and $24$ percent of this change is due to the
component of technical change that was sector-specific and skill-neutral,
and that this component served to affect the skill premium through
compositional changes. Moreover, this sector-specific skill-neutral
component of technical change was also responsible for all of the structural
change observed in the data. }

{\normalsize Our findings have important implications for predicting the
future evolution of the skill premium, since the continued growth of the
value-added share of the skill-intensive sector will exert upward pressure
on this premium even in the absence of skill-biased technological change. }

{\normalsize In order to best articulate the mechanism of skill-biased
structural change we have purposefully focused on a simple two-sector model.
There is good reason to think that the mechanism we have highlighted is also
at work at a more disaggregated level, so it is of interest to explore this
mechanism in a richer model. It would also be of interest to explicitly introduce capital and investment into the analysis.}

{\normalsize \bigskip }

{\normalsize \newpage }

{\normalsize 
\bibliographystyle{ecta}
\bibliography{references}
}

\newpage
\appendix
\section*{Appendix 1}
The results in Table 5 of the paper indicated that the contribution of
changes in the $A_{j}$ to the change in the skill premium were relatively
constant across the four different specifications corresponding to different
values for $\varepsilon $, the elasticity parameter in preferences. In this
appendix we examine the reasons behind this result.

As noted in the text, the dominant channel through which changes in the $%
A_{j}$ affect $w_{H}$ is through changes in the relative sectoral
expenditure shares. For this reason, we focus on why the contribution of the 
$A_{j}$ to changes in sectoral expenditure shares is very similar across
specifications.

Looking at this from the household perspective, the changes in expenditure
shares are determined by the change in income and the change in relative
prices. To understand the relative contributions of the three different
driving forces (the $f_{i}$, the $\alpha _{j}$ and the $A_{j}$) to changes
in relative expenditure shares it is therefore of interest to assess the
effect of each driving force on income and relative prices.

It is useful to start with an accounting decomposition. Let $\pi _{I}$ be
the share of sectoral reallocation accounted for by the change in income ($I$%
) and let $\pi _{P}$ be the share of sectoral allocation accounted for by
changes in the price of services relative to goods ($P$), so that $\pi
_{I}+\pi _{P}=1$. Importantly, the value of the $\pi _{j}$ will vary across
the four specifications that have different values of $\varepsilon $. Next,
let $\tau _{If},\tau _{I\alpha }$, and $\tau _{IA}$ denote the fraction of
the overall change in income accounted for by the change in each of the $%
f_{i}$, the $\alpha _{j}$ and the $A_{j}$ respectively. Similarly, let $\tau
_{Pf},\tau _{P\alpha }$, and $\tau _{PA}$ denote the fraction of the overall
change in \ relative prices accounted for by the change in each of the $f_{i}
$, the $\alpha _{j}$ and the $A_{j}$ respectively. By construction, $\tau
_{If}+\tau _{I\alpha }+\tau _{IA}=1$ and $\tau
_{Pf}+\tau _{P\alpha }+\tau _{PA}=1$. Thus, by
construction, it follows that:

\begin{equation*}
\pi _{I}(\tau _{If}+\tau _{I\alpha }+\tau _{IA})+\pi _{P}(\tau _{Pf}+\tau
_{P\alpha }+\tau _{PA})=1
\end{equation*}%
The fraction of overall sectoral reallocation that is accounted for by
changes in the $A_{j}$, which we denote by $S_{A}$, is then given by:

\begin{equation*}
S_{A}=\pi _{I}\tau _{IA}+\pi _{P}\tau _{IA}
\end{equation*}

As noted above, the values of $\pi _{I}$ and $\pi _{P}$ vary across the two
specifications, suggesting that $S_{A}$ might also vary across
specifications, though this will depend upon the values of $\tau _{IA}$ and $\tau _{PA}$  
and how they change across specifications.

Next we examine how each of the different subscripted $\pi$'s and $\tau$'s vary across
specifications. Table A1 shows the values for $\pi _{I}$ and $\pi _{P}$.

\begin{center}
\begin{tabular}{ccccc}
\multicolumn{5}{c}{Table A1} \\ 
\multicolumn{5}{c}{Values for $\pi _{I}$ and $\pi _{P}$} \\ \hline\hline
& $\varepsilon =.01$ & $\varepsilon =.10$ & $\varepsilon =.50$ & $%
\varepsilon =1.00$ \\ \hline\hline
$\pi _{I}$ & $.25$ & $.29$ & $.50$ & $1.00$ \\ 
$\pi _{P}$ & $.75$ & $.71$ & $.50$ & $0.00$ \\ \hline
\end{tabular}
\end{center}

Table A2 reports the information necessary to compute the various subscripted $\tau$'s. The
first row of this table shows the total change in income and the relative
price of services as we move from the initial equilibrium to the final
equilibrium. By virtue of our calibration procedure, these changes are the
same for each specification. The second row shows the effect on income and
the relative price of services when we move from the initial equilibrium to
the new equilibrium in which we change the value of the $f_{i}$ from their
initial value to their final value, holding the $\alpha _{j}$ and the $A_{j}$
constant. The third and fourth rows show in turn the effect of changes in
the $\alpha _{j}$ and the $A_{j}$, in each case starting from the
equilibrium corresponding to the second row of the table in which the $f_{i}$
take on their final values.

\begin{center}
\begin{tabular}{ccccccccc}
\multicolumn{9}{c}{Table A2} \\ 
\multicolumn{9}{c}{Contributions to Changes in Income and Relative Prices}
\\ \hline\hline
& \multicolumn{2}{c}{$\varepsilon =.01$} & \multicolumn{2}{c}{$\varepsilon
=.10$} & \multicolumn{2}{c}{$\varepsilon =.50$} & \multicolumn{2}{c}{$%
\varepsilon =1.00$} \\ \hline\hline
& $\Delta I$ & $\Delta \frac{p_{s}}{p_{g}}$ & $\Delta I$ & $\Delta \frac{%
p_{s}}{p_{g}}$ & $\Delta I$ & $\Delta \frac{p_{s}}{p_{g}}$ & $\Delta I$ & $%
\Delta \frac{p_{s}}{p_{g}}$ \\ 
{\small Total Change} & $.184$ & $.455$ & $.184$ & $.455$ & $.184$ & $.455$
& $.181$ & $.455$ \\ 
{\small Change from }$\Delta f_{i}$ & $.056$ & -$.131$ & $.056$ & -$.131$ & $%
.056$ & -$.131$ & $.54$ & -$.132$ \\ 
{\small Change from } $\Delta \alpha _{j}$ & -$.060$ & -$.004$ & -$.060$ & -$%
.005$ & -$.059$ & -$.011$ & -$.058$ & -$.022$ \\ 
{\small Change from }$\Delta A_{j}$ & $.200$ & $.631$ & $.200$ & $.631$ & $%
.201$ & $.635$ & $.203$ & $.646$ \\ \hline
\end{tabular}
\end{center}

The key takeaway from this table is that the contribution of each driving
force to both margins (income and prices) is approximately constant across
the four specifications, implying that the $\tau _{IA}$ and $\tau _{PA}$  are roughly
constant. In particular, $\tau _{IA}$ is roughly constant at $1.09$ and $%
\tau _{PA}$ is roughly constant at $1.38$. As noted above, $\pi _{I}$ varies
from .25 to 1.00 as $\varepsilon $ varies between $.01$ and $1.00$. It
follows that $S_{A}$ varies within the relatively small range of 1.30 to
1.09. The decrease from 1.30 to 1.09 represents a decrease of a bit more than 15\%. This is consistent with the results displayed in Table 5, which
show that the contribution of changes in the $A_{j}$ to the change in $w_{H}$
decreases from 18.1\% to 15.3\%, a drop of \ roughly 16\% as we move from $%
\varepsilon =.01$ to $\varepsilon =1.00$. Importantly, it is because the gap
between $\tau _{IA}$ and $\tau _{PA}$ is not that large that the variation
across specifications is relatively small.

The previous accounting decompositions raise two questions. First, why are
$\tau _{IA}$ and $\tau _{PA}$  both roughly constant? And second, why is the gap between $%
\tau _{IA}$ and $\tau _{PA}$ not that large? We deal with each of these in
turn.

We start with the first question, and proceed by considering each row in
Table A2. The first row indicates that the effect of a change in the $f_{i}$
on each of income and relative prices is roughly constant across the
specifications. We argue that this is effectively a result of our
calibration strategy. To see why, recall that in each specification we
choose a value for $\rho $ such that the implied effect of a change in the $%
f_{i}$ on $w_{H}$ is the same to first order. It follows that the effect of
the change in the $f_{i}$ on income are calibrated to be the same (to first
order) in each specification. Additionally, the first order effect of a
change in the $f_{i}$ on relative prices is only a function of the factor
shares. Because our calibration procedure necessarily matches the factor
payment shares in the data, it follows that changes in relative prices are
also the same to first order. In summary, the fact that the effect of
changes in the $f_{i}$ holding the $\alpha _{j}$ and the $A_{j}$ constant on
each of income and relative prices is roughly constant across the four
specifications is a feature of our calibration strategy.

Next we consider the effects due to a change in the $\alpha _{j}$. First we
consider the effects on income. To do this we ask how a change in $\alpha
_{j}$ affects output in sector $j$. To first order, one can show that the
effect is given by:

\begin{equation*}
\frac{dy_{j}}{y_{j}}=\frac{\rho }{\rho -1}\frac{(\theta _{j}-\alpha _{j})}{%
(1-\alpha _{j})}\frac{d\alpha _{j}}{\alpha _{j}}  \label{pacoequation}
\end{equation*}

Holding $\rho $ fixed, the effect of a given percentage change in $\alpha
_{j}$ on $y_{j}$ is dictated by factor shares. Our calibration procedure
necessarily matches the factor shares in all specifications, so that this
effect will be the same across specifications. However, as noted above, the
value of $\rho $ is not constant across specifications, and because the
value of $\rho $ affects the imputation of the changes in the $\alpha _{j}$,
it follows that both the value of $\rho $ and the percent change in the $%
\alpha _{j}$ will vary across the specifications as $\varepsilon $ varies.
But we argue that these effects are small. First, as $\varepsilon $ varies
from $.01$ to $.50$, the implied changes in $\rho $ are second order. But
second, and more generally, our procedure for computing the series for the $%
\alpha _{j}$ imposes that the following first order condition holds at each
point in time:

\begin{equation*}
\frac{H_{j}}{L_{j}}=\left(\frac{1}{w_{H}}\right)^{\rho }\left(\frac{\alpha _{j}}{1-\alpha
_{j}}\right)^{\rho }
\end{equation*}%
That is, any change in the value of $\rho $ will lead to a new series for $%
\alpha _{j}(1-\alpha _{j})$ that keeps the value of $[\alpha _{j}/(1-\alpha
_{j})]^{\rho }$ constant. And because this value remains constant across
specifications it follows that the effect on relative labor demands will
also be constant across specifications. Put somewhat differently,any change
in the value of $\rho $ will be offset by a corresponding change in the
implied series for $\alpha _{j}(1-\alpha _{j})$ that is neutral with respect
to the impact on relative demands for labor. We conclude that the effect of
the change in the $\alpha _{j}$ on income is roughly constant largely on
account of our calibration procedure.

To first order, the effect of the changes in the $\alpha _{j}$ on the
relative price of services depends on how the changes in the $\alpha _{j}$
affect relative costs of production across the two sectors. A given increase
in each of the $\alpha _{j}$ will have a larger impact on the costs in the
service sector given that high skill labor is a higher fraction of total
costs in that sector. However, our procedure implies that the increase in $%
\alpha _{G}$ is larger than the increase in $\alpha _{S}$, and roughly
offsets the first factor just mentioned. It follows that the effect of the
change in the $\alpha _{j}$ on relative prices is effectively zero. Across
specifications, we change both the implied changes in the $\alpha _{j}$ as
well as the effect of a given change in the $\alpha _{j}$ on costs. These
effects are roughy offsetting, which is why the overall result is
effectively constant across specifications. In summary, the fact that this
effect is zero is a feature of the data. Had the changes in the two $\alpha
_{j}$ been different, the overall effect would have not been zero.

To this point we have argued that the constancy of the effects of changes in
the $f_{i}$ and the $\alpha _{j}$ are largely a feature of our calibration
strategy and the procedure for measuring the $\alpha _{j}$. If these effects
are independent of the value of $\varepsilon $ and the total effect on
income and prices is the same for all values of $\varepsilon $, it follows
that the effect of the change in the $A_{j}$ must also be constant. So the
answer to the first question, concerning the relative constancy of $\tau
_{IA}$ and $\tau
_{PA}$ is that this is effectively an implication of our calibration
strategy, in that for each value of $\varepsilon $, the values of $\rho $
and $\bar{c}_{s}$ are adjusted so that both the pure effect of a change in
the $f_{i}$ on $w_{H}$ is constant across specifications, that the total
effect on both income and relative prices is the same across all
specifications, and that factor and expenditure shares are all the same
across all specifications.

Given that the values of $\tau _{IA}$ and $\tau _{PA}$ are roughly constant
and that $\pi _{I}$ varies from .25 to 1.00, the variation in $S_{A}$ will
be dictated by the gap between $\tau _{IA}$ and $\tau _{PA}$. As noted
above, we find that this gap is roughly $.30$, so that the variation in $%
S_{A}$ is roughly .20 as $\pi _{I}$ varies from .25 to 1.00. The fact that
the gap is equal to $.30$ is a feature of the data and not a feature of our
calibration.

\end{document}
